
% Copyright Glen Newton glen.newton@gmail.com
%
\documentclass[12pt]{article}
\usepackage{imakeidx}
\usepackage{hyperref}
\usepackage[most]{tcolorbox}
%\usepackage{savetrees}
\usepackage{graphicx}
\usepackage[utf8]{inputenc}
\usepackage[english]{babel}
\usepackage{multicol}
\usepackage{seqsplit}
\usepackage{xcolor}
\usepackage{etoolbox}
\usepackage{tocloft}
\usepackage{fancyhdr}
\usepackage{listings}
\usepackage{tabularx}
\usepackage[export]{adjustbox}

\usepackage[inner=10mm,outer=10mm]{geometry}


\usepackage{../sty/awsicons} 

\graphicspath{{../icons_tex/}}

%%%%%%%%%%%%%%%%%%%%%%%%%%%
\title{\LaTeX\ Style for AWS Service and Resource Icons}
\author{Glen Newton \\\texttt{\href{mailto:glen.newton@gmail.com}{glen.newton@gmail.com}}}

\date{\today}

%%%%%%%%%%%%%%%%%%%%%%%%%%%%%%%%%%%%%%%%%%%%%%%%%%%%%%%%%%%%%%%%%%%%%%%%
%% Change below lines to change look of document %%%%%%%%%%%%%%%%%%%%%%%
%%%%%%%%%%%%%%%%%%%%%%%%%%%%%%%%%%%%%%%%%%%%%%%%%%%%%%%%%%%%%%%%%%%%%%%%

\pagestyle{fancy}
\fancyhf{}
\rfoot{\hyperref[index]{General Index}}
\cfoot{\hyperref[macros]{Macro Index}}
\lfoot{\hyperref[toc]{TOC}}

\rhead{Page \thepage}
\chead{\href{https://github.com/gnewton/awsArchIcons2LaTeX}{awsArchIcons2LaTeX}}

\newcommand{\iconsize}{1cm}
\newcommand{\entrySpacing}{2mm}
\newcommand{\minipagespacing}{8cm}
%\renewcommand{\familydefault}{\sfdefault}

%% Arch and Res sections start/end
\newcommand{\archStart}{\section{Services}\begin{multicols}{2}\footnotesize}
\newcommand{\archEnd}{\end{multicols}\clearpage}
\newcommand{\resStart}{\newpage\section{Resources}\begin{multicols}{2}\footnotesize}
\newcommand{\resEnd}{\end{multicols}\clearpage}

%% Command used for each icon entry
%\newcommand{\gxs}[5]{\vcenteredinclude{#2}{\hspace{3mm} \begin{minipage}{\minipagespacing}{\bf \small{#1}}\\ {\footnotesize \tt \seqsplit{#3}} \\\tt Macro:\seqsplit{#4}\index[macros]{#4 (#5)} \end{minipage}}\vspace{\vspacing}\newline}

\newcommand{\gxs}[5]{
\noindent\begin{tabularx}{9cm}{p{9mm}p{8.1cm}}
               {#2}&
               #1\newline
               #3\newline
               Macro: #4 \index[macros]{#4 (#5)}\newline \\
\end{tabularx}
\vspace{\entrySpacing}
}

%% From: https://tex.stackexchange.com/a/847
\hypersetup{
    colorlinks,
    linkcolor={red!50!black},
    citecolor={blue!50!black},
    urlcolor={blue!80!black}
}

%%%%%%%%%%%%%%%%%%%%%%%%%%%%%%%%%%%%%%%%%%%%%%%%%%%%%%%%%%%%%%%%%%%%%%%%
%% Do not change rest of preamble (below this line) unless you know what
%% you are doing %%%%%%%%%%%%%%%%%%%%%%%%%%%%%%%%%%%%%%%%%%%%%%%%%%%%%%%
%%%%%%%%%%%%%%%%%%%%%%%%%%%%%%%%%%%%%%%%%%%%%%%%%%%%%%%%%%%%%%%%%%%%%%%%

%% From: https://codepunk.io/change-the-background-color-of-a-blockquote-in-latex-2/
\definecolor{block-gray}{gray}{.75}
\newtcolorbox{blockquote}{colback=block-gray,grow to right by=-1mm,grow to left by=-1mm,boxrule=0pt,boxsep=0pt,breakable}

\renewcommand{\cftsecleader}{\cftdotfill{\cftdotsep}}

\makeatletter
\appto{\endmulticols}{\@doendpe}
\makeatother

%% Ref to index: From https://tex.stackexchange.com/questions/182710/how-to-reference-the-index
\makeindex[intoc,title={Index: General\label{index}}]
\makeindex[name=macros, intoc,title={Index: \LaTeX \ \ Macros\label{macros}}]

%% "Too many open files" problem: From: https://stackoverflow.com/questions/1715677/error-too-many-open-files-in-pdflatex/1720556#1720556
\let\mypdfximage\pdfximage
\def\pdfximage{\immediate\mypdfximage}

 \newcommand{\vcenteredinclude}[1]{\begingroup
   \setbox0=\hbox{#1}%
   \parbox[t]{\wd0}{\box0}\endgroup}

%%%%%%%%%%%%%%%%%%%%%%%%%%%%%%%%%%%%%%%%%%%%%%%%%%%%%%%%%%%%%%%%%%%%%%%
\begin{document}

\maketitle
\tableofcontents\label{toc}
%\clearpage
\noindent\makebox[\linewidth]{\rule{\paperwidth}{0.4pt}}

%%%%%%
\section{What is this?}
\LaTeX\ style (\texttt{awsicons.sty}) which allows you to use the \href{https://aws.amazon.com/architecture/icons/}{AWS architectural icons} in your \LaTeX\  documents.

\vspace{3mm}

\noindent The AWS SVG icons are converted to {\LaTeX}--compatible PDF form using \href{https://inkscape.org/}{Inkscape} \texttt{\inkscapeVersion} using the purpose--written Go program \href{https://github.com/gnewton/awsArchIcons2LaTeX/aws2tex}{aws2tex}. 

\noindent This Go program, \texttt{aws2tex}, creates:
\begin{itemize}
\item \texttt{awsicons.sty} file
\item \LaTeX--compatible PDF versions of icons (and written to the \texttt{icons\_tex} directory, using Inkscape).
\item This document, which is to illustrate as well as index these icons and their corresponding \LaTeX\ \  macros.
  \newline
  \textbf{NB:} Not all icons are converted:
  \begin{itemize}
  \item For \textit{Resources}, only the \texttt{\_Light} versions. The \texttt{\_Dark} are omitted.
  \item For \textit{Services}, only the \texttt{\_64} versions. The lower resolution \texttt{\_16,\_32,\_48} are omitted.
  \end{itemize}
  
\noindent For instructions on how to generate this document with \textit{all} icons, please see the instructions in the \href{https://github.com/gnewton/awsArchIcons2LaTeX}{README.md} at the github repo.\\
This document is: \url{https://github.com/gnewton/awsArchIcons2LaTeX/awsAllIcons.pdf}
\end{itemize}


%%%%%%
\section{How to use it?}
Icons can be added to \LaTeX\ documents using the \LaTeX macros defined in the package, \texttt{awsicons.sty}. All you need to do is use the package AND add the directory containing all the gnerated icons to the \texttt{{\textbackslash}graphicspath\{...\}}.\\
For example:

\begin{blockquote}
\begin{verbatim}
% preamble
\usepackage{../sty/awsicons} 
\graphicspath{../icons_tex/}
% after \begin{document}
Here is a 1cm API Gateway:
\RAPIGatewayEndpoint{1cm}

Here is a 3cm Lambda function:
\RLambdaLambdaFunc{3cm}

\end{verbatim}
\end{blockquote}

\noindent Resulting in:\\
\begin{blockquote}
Here is a 1cm API Gateway:
\RAPIGatewayEndpoint{1cm}

Here is a 3cm Lambda function:
\RLambdaLambdaFunc{3cm}

\end{blockquote}

[Without the gray background]
    
%%%%%%
\subsection{Examples}
There are more complex examples in the \href{https://github.com/gnewton/awsArchIcons2LaTeX/tree/main/examples}{examples} directory, including one which recreates an AWS Architecture blog diagram.

\begin{itemize}
\item \href{https://github.com/gnewton/awsArchIcons2LaTeX/blob/main/examples/simple.pdf}{Simple example}
\item \textit{\href{https://github.com/gnewton/awsArchIcons2LaTeX/raw/main/examples/Data-pipeline-Grov-Technologies.pdf}{Building a Controlled Environment Agriculture Platform}\footnote{\url{https://aws.amazon.com/blogs/architecture/building-a-controlled-environment-agriculture-platform/}}}, by Ashu Joshi, 2020.12.22.
\end{itemize}
  
\section{Sources}
The AWS Architecture icons used here are from \url{https://aws.amazon.com/architecture/icons/}.\\
The specific assets package used: \href{https://d1.awsstatic.com/webteam/architecture-icons/Q32020/\assetZipFile}{\assetZipFile}

\vspace{3cm}
\noindent \copyright\ Glen Newton 2020\\
Icons in this document (I believe) are: \ \copyright\ Amazon Web Services (AWS)
\vspace{5mm}

\setlength{\parindent}{0pt}

%%%%%%%%%%%%% ICON CONTENT: generated by aws2tex:
\resStart
\gxs{\href{https://www.google.com/search?q=AWS+API+Gateway+Endpoint}{API Gateway: Endpoint} \index{API Gateway!Endpoint}\index{Endpoint (API Gateway)}} {\RAPIGatewayEndpoint{\iconsize}} {Res\_Amazon-API-Gateway\_Endpoint\_48\_Light.pdf} {{\textbackslash}RAPIGatewayEndpoint} {API Gateway: Endpoint}

\gxs{\href{https://www.google.com/search?q=AWS+App+Mesh+Mesh}{App Mesh Mesh} \index{App Mesh Mesh}} {\RAppMeshMesh{\iconsize}} {Res\_AWS-App-Mesh-Mesh\_48\_Light.pdf} {{\textbackslash}RAppMeshMesh} {App Mesh Mesh}

\gxs{\href{https://www.google.com/search?q=AWS+App+Mesh+Virtual+Node}{App Mesh Virtual Node} \index{App Mesh Virtual Node}} {\RAppMeshVirtNode{\iconsize}} {Res\_AWS-App-Mesh-Virtual-Node\_48\_Light.pdf} {{\textbackslash}RAppMeshVirtNode} {App Mesh Virtual Node}

\gxs{\href{https://www.google.com/search?q=AWS+App+Mesh+Virtual+Router}{App Mesh Virtual Router} \index{App Mesh Virtual Router}} {\RAppMeshVirtRouter{\iconsize}} {Res\_AWS-App-Mesh-Virtual-Router\_48\_Light.pdf} {{\textbackslash}RAppMeshVirtRouter} {App Mesh Virtual Router}

\gxs{\href{https://www.google.com/search?q=AWS+App+Mesh+Virtual+Service}{App Mesh Virtual Service} \index{App Mesh Virtual Service}} {\RAppMeshVirtSvc{\iconsize}} {Res\_AWS-App-Mesh-Virtual-Service\_48\_Light.pdf} {{\textbackslash}RAppMeshVirtSvc} {App Mesh Virtual Service}

\gxs{\href{https://www.google.com/search?q=AWS+App+Mesh+Virtual+Gateway}{App Mesh: Virtual Gateway} \index{App Mesh!Virtual Gateway}\index{Virtual Gateway (App Mesh)}} {\RAppMeshVirtGateway{\iconsize}} {Res\_App-Mesh\_Virtual-Gateway\_48\_Light.pdf} {{\textbackslash}RAppMeshVirtGateway} {App Mesh: Virtual Gateway}

\gxs{\href{https://www.google.com/search?q=AWS+Aurora+Amazon+Aurora+Instance+alternate}{Aurora: Amazon Aurora Instance alternate} \index{Aurora!Amazon Aurora Instance alternate}\index{Amazon Aurora Instance alternate (Aurora)}} {\RAurAurInstalternate{\iconsize}} {Res\_Amazon-Aurora\_Amazon-Aurora-Instance-alternate\_48\_Light.pdf} {{\textbackslash}RAurAurInstalternate} {Aurora: Amazon Aurora Instance alternate}

\gxs{\href{https://www.google.com/search?q=AWS+Aurora+Amazon+RDS+Instance}{Aurora: Amazon RDS Instance} \index{Aurora!Amazon RDS Instance}\index{Amazon RDS Instance (Aurora)}} {\RAurRDSInst{\iconsize}} {Res\_Amazon-Aurora\_Amazon-RDS-Instance\_48\_Light.pdf} {{\textbackslash}RAurRDSInst} {Aurora: Amazon RDS Instance}

\gxs{\href{https://www.google.com/search?q=AWS+Aurora+Amazon+RDS+Instance+Alternate}{Aurora: Amazon RDS Instance Alternate} \index{Aurora!Amazon RDS Instance Alternate}\index{Amazon RDS Instance Alternate (Aurora)}} {\RAurRDSInstAltern{\iconsize}} {Res\_Amazon-Aurora\_Amazon-RDS-Instance-Aternate\_48\_Light.pdf} {{\textbackslash}RAurRDSInstAltern} {Aurora: Amazon RDS Instance Alternate}

\gxs{\href{https://www.google.com/search?q=AWS+Aurora+Instance}{Aurora: Instance} \index{Aurora!Instance}\index{Instance (Aurora)}} {\RAurInst{\iconsize}} {Res\_Amazon-Aurora-Instance\_48\_Light.pdf} {{\textbackslash}RAurInst} {Aurora: Instance}

\gxs{\href{https://www.google.com/search?q=AWS+Aurora+MariaDB+Instance}{Aurora: MariaDB Instance} \index{Aurora!MariaDB Instance}\index{MariaDB Instance (Aurora)}} {\RAurMariaDBInst{\iconsize}} {Res\_Amazon-Aurora-MariaDB-Instance\_48\_Light.pdf} {{\textbackslash}RAurMariaDBInst} {Aurora: MariaDB Instance}

\gxs{\href{https://www.google.com/search?q=AWS+Aurora+MariaDB+Instance+Alternate}{Aurora: MariaDB Instance Alternate} \index{Aurora!MariaDB Instance Alternate}\index{MariaDB Instance Alternate (Aurora)}} {\RAurMariaDBInstAltern{\iconsize}} {Res\_Amazon-Aurora-MariaDB-Instance-Alternate\_48\_Light.pdf} {{\textbackslash}RAurMariaDBInstAltern} {Aurora: MariaDB Instance Alternate}

\gxs{\href{https://www.google.com/search?q=AWS+Aurora+MySQL+Instance}{Aurora: MySQL Instance} \index{Aurora!MySQL Instance}\index{MySQL Instance (Aurora)}} {\RAurMySQLInst{\iconsize}} {Res\_Amazon-Aurora-MySQL-Instance\_48\_Light.pdf} {{\textbackslash}RAurMySQLInst} {Aurora: MySQL Instance}

\gxs{\href{https://www.google.com/search?q=AWS+Aurora+MySQL+Instance+Alternate}{Aurora: MySQL Instance Alternate} \index{Aurora!MySQL Instance Alternate}\index{MySQL Instance Alternate (Aurora)}} {\RAurMySQLInstAltern{\iconsize}} {Res\_Amazon-Aurora-MySQL-Instance-Alternate\_48\_Light.pdf} {{\textbackslash}RAurMySQLInstAltern} {Aurora: MySQL Instance Alternate}

\gxs{\href{https://www.google.com/search?q=AWS+Aurora+Oracle+Instance}{Aurora: Oracle Instance} \index{Aurora!Oracle Instance}\index{Oracle Instance (Aurora)}} {\RAurOracleInst{\iconsize}} {Res\_Amazon-Aurora-Oracle-Instance\_48\_Light.pdf} {{\textbackslash}RAurOracleInst} {Aurora: Oracle Instance}

\gxs{\href{https://www.google.com/search?q=AWS+Aurora+Oracle+Instance+Alternate}{Aurora: Oracle Instance Alternate} \index{Aurora!Oracle Instance Alternate}\index{Oracle Instance Alternate (Aurora)}} {\RAurOracleInstAltern{\iconsize}} {Res\_Amazon-Aurora-Oracle-Instance-Alternate\_48\_Light.pdf} {{\textbackslash}RAurOracleInstAltern} {Aurora: Oracle Instance Alternate}

\gxs{\href{https://www.google.com/search?q=AWS+Aurora+PIOPS}{Aurora: PIOPS} \index{Aurora!PIOPS}\index{PIOPS (Aurora)}} {\RAurPIOPS{\iconsize}} {Res\_Amazon-Aurora-PIOPS\_48\_Light.pdf} {{\textbackslash}RAurPIOPS} {Aurora: PIOPS}

\gxs{\href{https://www.google.com/search?q=AWS+Aurora+PostgreSQL+Instance}{Aurora: PostgreSQL Instance} \index{Aurora!PostgreSQL Instance}\index{PostgreSQL Instance (Aurora)}} {\RAurPostgreSQLInst{\iconsize}} {Res\_Amazon-Aurora-PostgreSQL-Instance\_48\_Light.pdf} {{\textbackslash}RAurPostgreSQLInst} {Aurora: PostgreSQL Instance}

\gxs{\href{https://www.google.com/search?q=AWS+Aurora+PostgreSQL+Instance+Alternate}{Aurora: PostgreSQL Instance Alternate} \index{Aurora!PostgreSQL Instance Alternate}\index{PostgreSQL Instance Alternate (Aurora)}} {\RAurPostgreSQLInstAltern{\iconsize}} {Res\_Amazon-Aurora-PostgreSQL-Instance-Alternate\_48\_Light.pdf} {{\textbackslash}RAurPostgreSQLInstAltern} {Aurora: PostgreSQL Instance Alternate}

\gxs{\href{https://www.google.com/search?q=AWS+Aurora+SQL+Server+Instance}{Aurora: SQL Server Instance} \index{Aurora!SQL Server Instance}\index{SQL Server Instance (Aurora)}} {\RAurSQLServerInst{\iconsize}} {Res\_Amazon-Aurora-SQL-Server-Instance\_48\_Light.pdf} {{\textbackslash}RAurSQLServerInst} {Aurora: SQL Server Instance}

\gxs{\href{https://www.google.com/search?q=AWS+Aurora+SQL+Server+Instance+Alternate}{Aurora: SQL Server Instance Alternate} \index{Aurora!SQL Server Instance Alternate}\index{SQL Server Instance Alternate (Aurora)}} {\RAurSQLServerInstAltern{\iconsize}} {Res\_Amazon-Aurora-SQL-Server-Instance-Alternate\_48\_Light.pdf} {{\textbackslash}RAurSQLServerInstAltern} {Aurora: SQL Server Instance Alternate}

\gxs{\href{https://www.google.com/search?q=AWS+Certificate+Manager+Certificate+Authority}{Certificate Manager: Certificate Authority} \index{Certificate Manager!Certificate Authority}\index{Certificate Authority (Certificate Manager)}} {\RCertManCertAuth{\iconsize}} {Res\_AWS-Certificate-Manager\_Certificate-Authority\_48\_Light.pdf} {{\textbackslash}RCertManCertAuth} {Certificate Manager: Certificate Authority}

\gxs{\href{https://www.google.com/search?q=AWS+Cloud+Map+Namespace}{Cloud Map Namespace} \index{Cloud Map Namespace}} {\RCloudMapNamespace{\iconsize}} {Res\_AWS-Cloud-Map-Namespace\_48\_Light.pdf} {{\textbackslash}RCloudMapNamespace} {Cloud Map Namespace}

\gxs{\href{https://www.google.com/search?q=AWS+Cloud+Map+Resource}{Cloud Map Resource} \index{Cloud Map Resource}} {\RCloudMapRes{\iconsize}} {Res\_AWS-Cloud-Map-Resource\_48\_Light.pdf} {{\textbackslash}RCloudMapRes} {Cloud Map Resource}

\gxs{\href{https://www.google.com/search?q=AWS+Cloud+Map+Service}{Cloud Map: Service} \index{Cloud Map!Service}\index{Service (Cloud Map)}} {\RCloudMapSvc{\iconsize}} {Res\_AWS-Cloud-Map\_Service\_48\_Light.pdf} {{\textbackslash}RCloudMapSvc} {Cloud Map: Service}

\gxs{\href{https://www.google.com/search?q=AWS+Cloud9+Cloud9}{Cloud9: Cloud9} \index{Cloud9!Cloud9}\index{Cloud9 (Cloud9)}} {\RCloudNineCloudNine{\iconsize}} {Res\_AWS-Cloud9\_Cloud9\_48\_Light.pdf} {{\textbackslash}RCloudNineCloudNine} {Cloud9: Cloud9}

\gxs{\href{https://www.google.com/search?q=AWS+CloudFormation+Change+Set}{CloudFormation: Change Set} \index{CloudFormation!Change Set}\index{Change Set (CloudFormation)}} {\RCloudFormChangeSet{\iconsize}} {Res\_AWS-CloudFormation\_Change-Set\_48\_Light.pdf} {{\textbackslash}RCloudFormChangeSet} {CloudFormation: Change Set}

\gxs{\href{https://www.google.com/search?q=AWS+CloudFormation+Stack}{CloudFormation: Stack} \index{CloudFormation!Stack}\index{Stack (CloudFormation)}} {\RCloudFormStack{\iconsize}} {Res\_AWS-CloudFormation\_Stack\_48\_Light.pdf} {{\textbackslash}RCloudFormStack} {CloudFormation: Stack}

\gxs{\href{https://www.google.com/search?q=AWS+CloudFormation+Template}{CloudFormation: Template} \index{CloudFormation!Template}\index{Template (CloudFormation)}} {\RCloudFormTemplate{\iconsize}} {Res\_AWS-CloudFormation\_Template\_48\_Light.pdf} {{\textbackslash}RCloudFormTemplate} {CloudFormation: Template}

\gxs{\href{https://www.google.com/search?q=AWS+CloudFront+Download+Distribution}{CloudFront: Download Distribution} \index{CloudFront!Download Distribution}\index{Download Distribution (CloudFront)}} {\RCloudFrontDownloadDistr{\iconsize}} {Res\_Amazon-CloudFront\_Download-Distribution\_48\_Light.pdf} {{\textbackslash}RCloudFrontDownloadDistr} {CloudFront: Download Distribution}

\gxs{\href{https://www.google.com/search?q=AWS+CloudFront+Edge+Location}{CloudFront: Edge Location} \index{CloudFront!Edge Location}\index{Edge Location (CloudFront)}} {\RCloudFrontEdgeLocation{\iconsize}} {Res\_Amazon-CloudFront\_Edge-Location\_48\_Light.pdf} {{\textbackslash}RCloudFrontEdgeLocation} {CloudFront: Edge Location}

\gxs{\href{https://www.google.com/search?q=AWS+CloudFront+Streaming+Distribution}{CloudFront: Streaming Distribution} \index{CloudFront!Streaming Distribution}\index{Streaming Distribution (CloudFront)}} {\RCloudFrontStreamingDistr{\iconsize}} {Res\_Amazon-CloudFront\_Streaming-Distribution\_48\_Light.pdf} {{\textbackslash}RCloudFrontStreamingDistr} {CloudFront: Streaming Distribution}

\gxs{\href{https://www.google.com/search?q=AWS+CloudSearch+Search+Documents}{CloudSearch: Search Documents} \index{CloudSearch!Search Documents}\index{Search Documents (CloudSearch)}} {\RCloudSearchSearchDocuments{\iconsize}} {Res\_Amazon-CloudSearch\_Search-Documents\_48\_Light.pdf} {{\textbackslash}RCloudSearchSearchDocuments} {CloudSearch: Search Documents}

\gxs{\href{https://www.google.com/search?q=AWS+Cloudwatch+Alarm}{Cloudwatch: Alarm} \index{Cloudwatch!Alarm}\index{Alarm (Cloudwatch)}} {\RCloudwatchAlarm{\iconsize}} {Res\_Amazon-Cloudwatch\_Alarm\_48\_Light.pdf} {{\textbackslash}RCloudwatchAlarm} {Cloudwatch: Alarm}

\gxs{\href{https://www.google.com/search?q=AWS+Cloudwatch+Event+Based}{Cloudwatch: Event Based} \index{Cloudwatch!Event Based}\index{Event Based (Cloudwatch)}} {\RCloudwatchEventBased{\iconsize}} {Res\_Amazon-Cloudwatch\_Event-Based\_48\_Light.pdf} {{\textbackslash}RCloudwatchEventBased} {Cloudwatch: Event Based}

\gxs{\href{https://www.google.com/search?q=AWS+Cloudwatch+Event+Time+Based}{Cloudwatch: Event Time Based} \index{Cloudwatch!Event Time Based}\index{Event Time Based (Cloudwatch)}} {\RCloudwatchEventTimeBased{\iconsize}} {Res\_Amazon-Cloudwatch\_Event-Time-Based\_48\_Light.pdf} {{\textbackslash}RCloudwatchEventTimeBased} {Cloudwatch: Event Time Based}

\gxs{\href{https://www.google.com/search?q=AWS+Cloudwatch+Logs}{Cloudwatch: Logs} \index{Cloudwatch!Logs}\index{Logs (Cloudwatch)}} {\RCloudwatchLogs{\iconsize}} {Res\_Cloudwatch\_Logs\_48\_Light.pdf} {{\textbackslash}RCloudwatchLogs} {Cloudwatch: Logs}

\gxs{\href{https://www.google.com/search?q=AWS+Cloudwatch+Rule}{Cloudwatch: Rule} \index{Cloudwatch!Rule}\index{Rule (Cloudwatch)}} {\RCloudwatchRule{\iconsize}} {Res\_Amazon-Cloudwatch\_Rule\_48\_Light.pdf} {{\textbackslash}RCloudwatchRule} {Cloudwatch: Rule}

\gxs{\href{https://www.google.com/search?q=AWS+Database+Migration+Service+Database+migration+workflow+job}{Database Migration Service: Database migration workflow job} \index{Database Migration Service!Database migration workflow job}\index{Database migration workflow job (Database Migration Service)}} {\RDBMigratSvcDBmigrationworkflowjob{\iconsize}} {Res\_AWS-Database-Migration-Service\_Database-migration-workflow-job\_48\_Light.pdf} {{\textbackslash}RDBMigratSvcDBmigrationworkflowjob} {Database Migration Service: Database migration workflow job}

\gxs{\href{https://www.google.com/search?q=AWS+Datasync+Agent}{Datasync: Agent} \index{Datasync!Agent}\index{Agent (Datasync)}} {\RDatasyncAgent{\iconsize}} {Res\_AWS-Datasync\_Agent\_48\_Light.pdf} {{\textbackslash}RDatasyncAgent} {Datasync: Agent}

\gxs{\href{https://www.google.com/search?q=AWS+Direct+Connect+Gateway}{Direct Connect Gateway} \index{Direct Connect Gateway}} {\RDirectConnectGateway{\iconsize}} {Res\_AWS-Direct-Connect-Gateway\_48\_Light.pdf} {{\textbackslash}RDirectConnectGateway} {Direct Connect Gateway}

\gxs{\href{https://www.google.com/search?q=AWS+Directory+Service+AD+Connector}{Directory Service: AD Connector} \index{Directory Service!AD Connector}\index{AD Connector (Directory Service)}} {\RDirSvcADConnector{\iconsize}} {Res\_AWS-Directory-Service\_AD-Connector\_48\_Light.pdf} {{\textbackslash}RDirSvcADConnector} {Directory Service: AD Connector}

\gxs{\href{https://www.google.com/search?q=AWS+Directory+Service+AWS+Managed+Microsoft+AD}{Directory Service: AWS Managed Microsoft AD} \index{Directory Service!AWS Managed Microsoft AD}\index{AWS Managed Microsoft AD (Directory Service)}} {\RDirSvcMngdMSAD{\iconsize}} {Res\_AWS-Directory-Service\_AWS-Managed-Microsoft-AD\_48\_Light.pdf} {{\textbackslash}RDirSvcMngdMSAD} {Directory Service: AWS Managed Microsoft AD}

\gxs{\href{https://www.google.com/search?q=AWS+Directory+Service+Simple+AD}{Directory Service: Simple AD} \index{Directory Service!Simple AD}\index{Simple AD (Directory Service)}} {\RDirSvcSimpleAD{\iconsize}} {Res\_AWS-Directory-Service\_Simple-AD\_48\_Light.pdf} {{\textbackslash}RDirSvcSimpleAD} {Directory Service: Simple AD}

\gxs{\href{https://www.google.com/search?q=AWS+DynamoDB+Amazon+DynamoDB+Accelerator}{DynamoDB: Amazon DynamoDB Accelerator} \index{DynamoDB!Amazon DynamoDB Accelerator}\index{Amazon DynamoDB Accelerator (DynamoDB)}} {\RDDBDDBAccelerator{\iconsize}} {Res\_Amazon-DynamoDB\_Amazon-DynamoDB-Accelerator\_48\_Light.pdf} {{\textbackslash}RDDBDDBAccelerator} {DynamoDB: Amazon DynamoDB Accelerator}

\gxs{\href{https://www.google.com/search?q=AWS+DynamoDB+Attribute}{DynamoDB: Attribute} \index{DynamoDB!Attribute}\index{Attribute (DynamoDB)}} {\RDDBAtt{\iconsize}} {Res\_Amazon-DynamoDB\_Attribute\_48\_Light.pdf} {{\textbackslash}RDDBAtt} {DynamoDB: Attribute}

\gxs{\href{https://www.google.com/search?q=AWS+DynamoDB+Attributes}{DynamoDB: Attributes} \index{DynamoDB!Attributes}\index{Attributes (DynamoDB)}} {\RDDBAtts{\iconsize}} {Res\_Amazon-DynamoDB\_Attributes\_48\_Light.pdf} {{\textbackslash}RDDBAtts} {DynamoDB: Attributes}

\gxs{\href{https://www.google.com/search?q=AWS+DynamoDB+Global+secondary+index}{DynamoDB: Global secondary index} \index{DynamoDB!Global secondary index}\index{Global secondary index (DynamoDB)}} {\RDDBGlblsecondaryindex{\iconsize}} {Res\_Amazon-DynamoDB\_Global-secondary-index\_48\_Light.pdf} {{\textbackslash}RDDBGlblsecondaryindex} {DynamoDB: Global secondary index}

\gxs{\href{https://www.google.com/search?q=AWS+DynamoDB+Item}{DynamoDB: Item} \index{DynamoDB!Item}\index{Item (DynamoDB)}} {\RDDBItem{\iconsize}} {Res\_Amazon-DynamoDB\_Item\_48\_Light.pdf} {{\textbackslash}RDDBItem} {DynamoDB: Item}

\gxs{\href{https://www.google.com/search?q=AWS+DynamoDB+Items}{DynamoDB: Items} \index{DynamoDB!Items}\index{Items (DynamoDB)}} {\RDDBItems{\iconsize}} {Res\_Amazon-DynamoDB\_Items\_48\_Light.pdf} {{\textbackslash}RDDBItems} {DynamoDB: Items}

\gxs{\href{https://www.google.com/search?q=AWS+DynamoDB+Table}{DynamoDB: Table} \index{DynamoDB!Table}\index{Table (DynamoDB)}} {\RDDBTable{\iconsize}} {Res\_Amazon-DynamoDB\_Table\_48\_Light.pdf} {{\textbackslash}RDDBTable} {DynamoDB: Table}

\gxs{\href{https://www.google.com/search?q=AWS+EC2+A1+Instance}{EC2: A1 Instance} \index{EC2!A1 Instance}\index{A1 Instance (EC2)}} {\RECTwoAOneInst{\iconsize}} {Res\_Amazon-EC2\_A1-Instance\_48\_Light.pdf} {{\textbackslash}RECTwoAOneInst} {EC2: A1 Instance}

\gxs{\href{https://www.google.com/search?q=AWS+EC2+AMI+Resource}{EC2: AMI: Resource} \index{EC2!AMI!Resource}\index{Resource (EC2 AMI)}} {\RECTwoAMIRes{\iconsize}} {Res\_Amazon-EC2\_AMI\_Resource\_48\_Light.pdf} {{\textbackslash}RECTwoAMIRes} {EC2: AMI: Resource}

\gxs{\href{https://www.google.com/search?q=AWS+EC2+AWS+Inferentia}{EC2: AWS Inferentia} \index{EC2!AWS Inferentia}\index{AWS Inferentia (EC2)}} {\RECTwoInferentia{\iconsize}} {Res\_Amazon-EC2\_AWS-Inferentia\_48\_Light.pdf} {{\textbackslash}RECTwoInferentia} {EC2: AWS Inferentia}

\gxs{\href{https://www.google.com/search?q=AWS+EC2+Auto+Scaling}{EC2: Auto Scaling} \index{EC2!Auto Scaling}\index{Auto Scaling (EC2)}} {\RECTwoAutoScaling{\iconsize}} {Res\_Amazon-EC2\_Auto-Scaling\_48\_Light.pdf} {{\textbackslash}RECTwoAutoScaling} {EC2: Auto Scaling}

\gxs{\href{https://www.google.com/search?q=AWS+EC2+C4+Instance}{EC2: C4 Instance} \index{EC2!C4 Instance}\index{C4 Instance (EC2)}} {\RECTwoCFourInst{\iconsize}} {Res\_Amazon-EC2\_C4-Instance\_48\_Light.pdf} {{\textbackslash}RECTwoCFourInst} {EC2: C4 Instance}

\gxs{\href{https://www.google.com/search?q=AWS+EC2+C5+Instance}{EC2: C5 Instance} \index{EC2!C5 Instance}\index{C5 Instance (EC2)}} {\RECTwoCFiveInst{\iconsize}} {Res\_Amazon-EC2\_C5-Instance\_48\_Light.pdf} {{\textbackslash}RECTwoCFiveInst} {EC2: C5 Instance}

\gxs{\href{https://www.google.com/search?q=AWS+EC2+C5n+Instance}{EC2: C5n Instance} \index{EC2!C5n Instance}\index{C5n Instance (EC2)}} {\RECTwoCFivenInst{\iconsize}} {Res\_Amazon-EC2\_C5n-Instance\_48\_Light.pdf} {{\textbackslash}RECTwoCFivenInst} {EC2: C5n Instance}

\gxs{\href{https://www.google.com/search?q=AWS+EC2+C6g+Instance}{EC2: C6g Instance} \index{EC2!C6g Instance}\index{C6g Instance (EC2)}} {\RECTwoCSixgInst{\iconsize}} {Res\_Amazon-EC2\_C6g-Instance\_48\_Light.pdf} {{\textbackslash}RECTwoCSixgInst} {EC2: C6g Instance}

\gxs{\href{https://www.google.com/search?q=AWS+EC2+D2+Instance}{EC2: D2 Instance} \index{EC2!D2 Instance}\index{D2 Instance (EC2)}} {\RECTwoDTwoInst{\iconsize}} {Res\_Amazon-EC2\_D2-Instance\_48\_Light.pdf} {{\textbackslash}RECTwoDTwoInst} {EC2: D2 Instance}

\gxs{\href{https://www.google.com/search?q=AWS+EC2+DB+Instance}{EC2: DB Instance} \index{EC2!DB Instance}\index{DB Instance (EC2)}} {\RECTwoDBInst{\iconsize}} {Res\_Amazon-EC2\_DB-Instance\_48\_Light.pdf} {{\textbackslash}RECTwoDBInst} {EC2: DB Instance}

\gxs{\href{https://www.google.com/search?q=AWS+EC2+Elastic+IP+Address}{EC2: Elastic IP Address} \index{EC2!Elastic IP Address}\index{Elastic IP Address (EC2)}} {\RECTwoElasticIPAddress{\iconsize}} {Res\_Amazon-EC2\_Elastic-IP-Address\_48\_Light.pdf} {{\textbackslash}RECTwoElasticIPAddress} {EC2: Elastic IP Address}

\gxs{\href{https://www.google.com/search?q=AWS+EC2+F1+Instance}{EC2: F1 Instance} \index{EC2!F1 Instance}\index{F1 Instance (EC2)}} {\RECTwoFOneInst{\iconsize}} {Res\_Amazon-EC2\_F1-Instance\_48\_Light.pdf} {{\textbackslash}RECTwoFOneInst} {EC2: F1 Instance}

\gxs{\href{https://www.google.com/search?q=AWS+EC2+G3+Instance}{EC2: G3 Instance} \index{EC2!G3 Instance}\index{G3 Instance (EC2)}} {\RECTwoGThreeInst{\iconsize}} {Res\_Amazon-EC2\_G3-Instance\_48\_Light.pdf} {{\textbackslash}RECTwoGThreeInst} {EC2: G3 Instance}

\gxs{\href{https://www.google.com/search?q=AWS+EC2+H1+Instance}{EC2: H1 Instance} \index{EC2!H1 Instance}\index{H1 Instance (EC2)}} {\RECTwoHOneInst{\iconsize}} {Res\_Amazon-EC2\_H1-Instance\_48\_Light.pdf} {{\textbackslash}RECTwoHOneInst} {EC2: H1 Instance}

\gxs{\href{https://www.google.com/search?q=AWS+EC2+HMI+Instance}{EC2: HMI Instance} \index{EC2!HMI Instance}\index{HMI Instance (EC2)}} {\RECTwoHMIInst{\iconsize}} {Res\_Amazon-EC2\_HMI-Instance\_48\_Light.pdf} {{\textbackslash}RECTwoHMIInst} {EC2: HMI Instance}

\gxs{\href{https://www.google.com/search?q=AWS+EC2+I3+Instance}{EC2: I3 Instance} \index{EC2!I3 Instance}\index{I3 Instance (EC2)}} {\RECTwoIThreeInst{\iconsize}} {Res\_Amazon-EC2\_I3-Instance\_48\_Light.pdf} {{\textbackslash}RECTwoIThreeInst} {EC2: I3 Instance}

\gxs{\href{https://www.google.com/search?q=AWS+EC2+Instance}{EC2: Instance} \index{EC2!Instance}\index{Instance (EC2)}} {\RECTwoInst{\iconsize}} {Res\_Amazon-EC2\_Instance\_48\_Light.pdf} {{\textbackslash}RECTwoInst} {EC2: Instance}

\gxs{\href{https://www.google.com/search?q=AWS+EC2+Instance+with+CloudWatch}{EC2: Instance with CloudWatch} \index{EC2!Instance with CloudWatch}\index{Instance with CloudWatch (EC2)}} {\RECTwoInstwithCloudWatch{\iconsize}} {Res\_Amazon-EC2\_Instance-with-CloudWatch\_48\_Light.pdf} {{\textbackslash}RECTwoInstwithCloudWatch} {EC2: Instance with CloudWatch}

\gxs{\href{https://www.google.com/search?q=AWS+EC2+Instances}{EC2: Instances} \index{EC2!Instances}\index{Instances (EC2)}} {\RECTwoInsts{\iconsize}} {Res\_Amazon-EC2\_Instances\_48\_Light.pdf} {{\textbackslash}RECTwoInsts} {EC2: Instances}

\gxs{\href{https://www.google.com/search?q=AWS+EC2+M4+Instance}{EC2: M4 Instance} \index{EC2!M4 Instance}\index{M4 Instance (EC2)}} {\RECTwoMFourInst{\iconsize}} {Res\_Amazon-EC2\_M4-Instance\_48\_Light.pdf} {{\textbackslash}RECTwoMFourInst} {EC2: M4 Instance}

\gxs{\href{https://www.google.com/search?q=AWS+EC2+M5+Instance}{EC2: M5 Instance} \index{EC2!M5 Instance}\index{M5 Instance (EC2)}} {\RECTwoMFiveInst{\iconsize}} {Res\_Amazon-EC2\_M5-Instance\_48\_Light.pdf} {{\textbackslash}RECTwoMFiveInst} {EC2: M5 Instance}

\gxs{\href{https://www.google.com/search?q=AWS+EC2+M5a+Instance}{EC2: M5a Instance} \index{EC2!M5a Instance}\index{M5a Instance (EC2)}} {\RECTwoMFiveaInst{\iconsize}} {Res\_Amazon-EC2\_M5a-Instance\_48\_Light.pdf} {{\textbackslash}RECTwoMFiveaInst} {EC2: M5a Instance}

\gxs{\href{https://www.google.com/search?q=AWS+EC2+M5n+Instance}{EC2: M5n Instance} \index{EC2!M5n Instance}\index{M5n Instance (EC2)}} {\RECTwoMFivenInst{\iconsize}} {Res\_Amazon-EC2\_M5n-Instance\_48\_Light.pdf} {{\textbackslash}RECTwoMFivenInst} {EC2: M5n Instance}

\gxs{\href{https://www.google.com/search?q=AWS+EC2+M6g+Instance}{EC2: M6g Instance} \index{EC2!M6g Instance}\index{M6g Instance (EC2)}} {\RECTwoMSixgInst{\iconsize}} {Res\_Amazon-EC2\_M6g-Instance\_48\_Light.pdf} {{\textbackslash}RECTwoMSixgInst} {EC2: M6g Instance}

\gxs{\href{https://www.google.com/search?q=AWS+EC2+P2+Instance}{EC2: P2 Instance} \index{EC2!P2 Instance}\index{P2 Instance (EC2)}} {\RECTwoPTwoInst{\iconsize}} {Res\_Amazon-EC2\_P2-Instance\_48\_Light.pdf} {{\textbackslash}RECTwoPTwoInst} {EC2: P2 Instance}

\gxs{\href{https://www.google.com/search?q=AWS+EC2+P3+Instance}{EC2: P3 Instance} \index{EC2!P3 Instance}\index{P3 Instance (EC2)}} {\RECTwoPThreeInst{\iconsize}} {Res\_Amazon-EC2\_P3-Instance\_48\_Light.pdf} {{\textbackslash}RECTwoPThreeInst} {EC2: P3 Instance}

\gxs{\href{https://www.google.com/search?q=AWS+EC2+R4+Instance}{EC2: R4 Instance} \index{EC2!R4 Instance}\index{R4 Instance (EC2)}} {\RECTwoRFourInst{\iconsize}} {Res\_Amazon-EC2\_R4-Instance\_48\_Light.pdf} {{\textbackslash}RECTwoRFourInst} {EC2: R4 Instance}

\gxs{\href{https://www.google.com/search?q=AWS+EC2+R5+Instance}{EC2: R5 Instance} \index{EC2!R5 Instance}\index{R5 Instance (EC2)}} {\RECTwoRFiveInst{\iconsize}} {Res\_Amazon-EC2\_R5-Instance\_48\_Light.pdf} {{\textbackslash}RECTwoRFiveInst} {EC2: R5 Instance}

\gxs{\href{https://www.google.com/search?q=AWS+EC2+R5a+Instance}{EC2: R5a Instance} \index{EC2!R5a Instance}\index{R5a Instance (EC2)}} {\RECTwoRFiveaInst{\iconsize}} {Res\_Amazon-EC2\_R5a-Instance\_48\_Light.pdf} {{\textbackslash}RECTwoRFiveaInst} {EC2: R5a Instance}

\gxs{\href{https://www.google.com/search?q=AWS+EC2+R5n+Instance}{EC2: R5n Instance} \index{EC2!R5n Instance}\index{R5n Instance (EC2)}} {\RECTwoRFivenInst{\iconsize}} {Res\_Amazon-EC2\_R5n-Instance\_48\_Light.pdf} {{\textbackslash}RECTwoRFivenInst} {EC2: R5n Instance}

\gxs{\href{https://www.google.com/search?q=AWS+EC2+R6g+Instance}{EC2: R6g Instance} \index{EC2!R6g Instance}\index{R6g Instance (EC2)}} {\RECTwoRSixgInst{\iconsize}} {Res\_Amazon-EC2\_R6g-Instance\_48\_Light.pdf} {{\textbackslash}RECTwoRSixgInst} {EC2: R6g Instance}

\gxs{\href{https://www.google.com/search?q=AWS+EC2+Rescue}{EC2: Rescue} \index{EC2!Rescue}\index{Rescue (EC2)}} {\RECTwoRescue{\iconsize}} {Res\_Amazon-EC2\_Rescue\_48\_Light.pdf} {{\textbackslash}RECTwoRescue} {EC2: Rescue}

\gxs{\href{https://www.google.com/search?q=AWS+EC2+Spot+Instance}{EC2: Spot Instance} \index{EC2!Spot Instance}\index{Spot Instance (EC2)}} {\RECTwoSpotInst{\iconsize}} {Res\_Amazon-EC2\_Spot-Instance\_48\_Light.pdf} {{\textbackslash}RECTwoSpotInst} {EC2: Spot Instance}

\gxs{\href{https://www.google.com/search?q=AWS+EC2+T2+Instance}{EC2: T2 Instance} \index{EC2!T2 Instance}\index{T2 Instance (EC2)}} {\RECTwoTTwoInst{\iconsize}} {Res\_Amazon-EC2\_T2-Instance\_48\_Light.pdf} {{\textbackslash}RECTwoTTwoInst} {EC2: T2 Instance}

\gxs{\href{https://www.google.com/search?q=AWS+EC2+T3+Instance}{EC2: T3 Instance} \index{EC2!T3 Instance}\index{T3 Instance (EC2)}} {\RECTwoTThreeInst{\iconsize}} {Res\_Amazon-EC2\_T3-Instance\_48\_Light.pdf} {{\textbackslash}RECTwoTThreeInst} {EC2: T3 Instance}

\gxs{\href{https://www.google.com/search?q=AWS+EC2+X1+Instance}{EC2: X1 Instance} \index{EC2!X1 Instance}\index{X1 Instance (EC2)}} {\RECTwoXOneInst{\iconsize}} {Res\_Amazon-EC2\_X1-Instance\_48\_Light.pdf} {{\textbackslash}RECTwoXOneInst} {EC2: X1 Instance}

\gxs{\href{https://www.google.com/search?q=AWS+EC2+X1e+Instance}{EC2: X1e Instance} \index{EC2!X1e Instance}\index{X1e Instance (EC2)}} {\RECTwoXOneeInst{\iconsize}} {Res\_Amazon-EC2\_X1e-Instance\_48\_Light.pdf} {{\textbackslash}RECTwoXOneeInst} {EC2: X1e Instance}

\gxs{\href{https://www.google.com/search?q=AWS+EC2+z1d+Instance}{EC2: z1d Instance} \index{EC2!z1d Instance}\index{z1d Instance (EC2)}} {\RECTwozOnedInst{\iconsize}} {Res\_Amazon-EC2\_z1d-Instance\_48\_Light.pdf} {{\textbackslash}RECTwozOnedInst} {EC2: z1d Instance}

\gxs{\href{https://www.google.com/search?q=AWS+EMR+Cluster}{EMR: Cluster} \index{EMR!Cluster}\index{Cluster (EMR)}} {\REMRCluster{\iconsize}} {Res\_Amazon-EMR\_Cluster\_48\_Light.pdf} {{\textbackslash}REMRCluster} {EMR: Cluster}

\gxs{\href{https://www.google.com/search?q=AWS+EMR+EMR+Engine}{EMR: EMR Engine} \index{EMR!EMR Engine}\index{EMR Engine (EMR)}} {\REMREMREngine{\iconsize}} {Res\_Amazon-EMR\_EMR-Engine\_48\_Light.pdf} {{\textbackslash}REMREMREngine} {EMR: EMR Engine}

\gxs{\href{https://www.google.com/search?q=AWS+EMR+EMR+Engine+MapR+M3}{EMR: EMR Engine MapR M3} \index{EMR!EMR Engine MapR M3}\index{EMR Engine MapR M3 (EMR)}} {\REMREMREngineMapRMThree{\iconsize}} {Res\_Amazon-EMR\_EMR-Engine-MapR-M3\_48\_Light.pdf} {{\textbackslash}REMREMREngineMapRMThree} {EMR: EMR Engine MapR M3}

\gxs{\href{https://www.google.com/search?q=AWS+EMR+EMR+Engine+MapR+M5}{EMR: EMR Engine MapR M5} \index{EMR!EMR Engine MapR M5}\index{EMR Engine MapR M5 (EMR)}} {\REMREMREngineMapRMFive{\iconsize}} {Res\_Amazon-EMR\_EMR-Engine-MapR-M5\_48\_Light.pdf} {{\textbackslash}REMREMREngineMapRMFive} {EMR: EMR Engine MapR M5}

\gxs{\href{https://www.google.com/search?q=AWS+EMR+EMR+Engine+MapR+M7}{EMR: EMR Engine MapR M7} \index{EMR!EMR Engine MapR M7}\index{EMR Engine MapR M7 (EMR)}} {\REMREMREngineMapRMSeven{\iconsize}} {Res\_Amazon-EMR\_EMR-Engine-MapR-M7\_48\_Light.pdf} {{\textbackslash}REMREMREngineMapRMSeven} {EMR: EMR Engine MapR M7}

\gxs{\href{https://www.google.com/search?q=AWS+EMR+HDFS+Cluster}{EMR: HDFS Cluster} \index{EMR!HDFS Cluster}\index{HDFS Cluster (EMR)}} {\REMRHDFSCluster{\iconsize}} {Res\_Amazon-EMR\_HDFS-Cluster\_48\_Light.pdf} {{\textbackslash}REMRHDFSCluster} {EMR: HDFS Cluster}

\gxs{\href{https://www.google.com/search?q=AWS+ElastiCache+Cache+Node}{ElastiCache: Cache Node} \index{ElastiCache!Cache Node}\index{Cache Node (ElastiCache)}} {\RElastiCacheCacheNode{\iconsize}} {Res\_Amazon-ElastiCache\_Cache-Node\_48\_Light.pdf} {{\textbackslash}RElastiCacheCacheNode} {ElastiCache: Cache Node}

\gxs{\href{https://www.google.com/search?q=AWS+ElastiCache+ElastiCache+for+Memcached}{ElastiCache: ElastiCache for Memcached} \index{ElastiCache!ElastiCache for Memcached}\index{ElastiCache for Memcached (ElastiCache)}} {\RElastiCacheElastiCacheforMemcached{\iconsize}} {Res\_Amazon-ElastiCache\_ElastiCache-for-Memcached\_48\_Light.pdf} {{\textbackslash}RElastiCacheElastiCacheforMemcached} {ElastiCache: ElastiCache for Memcached}

\gxs{\href{https://www.google.com/search?q=AWS+ElastiCache+ElastiCache+for+Redis}{ElastiCache: ElastiCache for Redis} \index{ElastiCache!ElastiCache for Redis}\index{ElastiCache for Redis (ElastiCache)}} {\RElastiCacheElastiCacheforRedis{\iconsize}} {Res\_Amazon-ElastiCache\_ElastiCache-for-Redis\_48\_Light.pdf} {{\textbackslash}RElastiCacheElastiCacheforRedis} {ElastiCache: ElastiCache for Redis}

\gxs{\href{https://www.google.com/search?q=AWS+Elastic+Beanstalk+Application}{Elastic Beanstalk: Application} \index{Elastic Beanstalk!Application}\index{Application (Elastic Beanstalk)}} {\REBApp{\iconsize}} {Res\_Amazon-Elastic-Beanstalk\_Application\_48\_Light.pdf} {{\textbackslash}REBApp} {Elastic Beanstalk: Application}

\gxs{\href{https://www.google.com/search?q=AWS+Elastic+Beanstalk+Deployment}{Elastic Beanstalk: Deployment} \index{Elastic Beanstalk!Deployment}\index{Deployment (Elastic Beanstalk)}} {\REBDeployment{\iconsize}} {Res\_Amazon-Elastic-Beanstalk\_Deployment\_48\_Light.pdf} {{\textbackslash}REBDeployment} {Elastic Beanstalk: Deployment}

\gxs{\href{https://www.google.com/search?q=AWS+Elastic+Block+Store+Multiple+Volumes}{Elastic Block Store: Multiple Volumes} \index{Elastic Block Store!Multiple Volumes}\index{Multiple Volumes (Elastic Block Store)}} {\REBSMultVols{\iconsize}} {Res\_Amazon-Elastic-Block-Store\_Multiple-Volumes\_48\_Light.pdf} {{\textbackslash}REBSMultVols} {Elastic Block Store: Multiple Volumes}

\gxs{\href{https://www.google.com/search?q=AWS+Elastic+Block+Store+Snapshot}{Elastic Block Store: Snapshot} \index{Elastic Block Store!Snapshot}\index{Snapshot (Elastic Block Store)}} {\REBSSnapshot{\iconsize}} {Res\_Amazon-Elastic-Block-Store\_Snapshot\_48\_Light.pdf} {{\textbackslash}REBSSnapshot} {Elastic Block Store: Snapshot}

\gxs{\href{https://www.google.com/search?q=AWS+Elastic+Block+Store+Volume}{Elastic Block Store: Volume} \index{Elastic Block Store!Volume}\index{Volume (Elastic Block Store)}} {\REBSVol{\iconsize}} {Res\_Amazon-Elastic-Block-Store\_Volume\_48\_Light.pdf} {{\textbackslash}REBSVol} {Elastic Block Store: Volume}

\gxs{\href{https://www.google.com/search?q=AWS+Elastic+Container+Registry+Image}{Elastic Container Registry: Image} \index{Elastic Container Registry!Image}\index{Image (Elastic Container Registry)}} {\RECRImg{\iconsize}} {Res\_Amazon-Elastic-Container-Registry\_Image\_48\_Light.pdf} {{\textbackslash}RECRImg} {Elastic Container Registry: Image}

\gxs{\href{https://www.google.com/search?q=AWS+Elastic+Container+Registry+Registry}{Elastic Container Registry: Registry} \index{Elastic Container Registry!Registry}\index{Registry (Elastic Container Registry)}} {\RECRReg{\iconsize}} {Res\_Amazon-Elastic-Container-Registry\_Registry\_48\_Light.pdf} {{\textbackslash}RECRReg} {Elastic Container Registry: Registry}

\gxs{\href{https://www.google.com/search?q=AWS+Elastic+Container+Service+Container+1}{Elastic Container Service: Container 1} \index{Elastic Container Service!Container 1}\index{Container 1 (Elastic Container Service)}} {\RECSContainerOne{\iconsize}} {Res\_Amazon-Elastic-Container-Service\_Container-1\_48\_Light.pdf} {{\textbackslash}RECSContainerOne} {Elastic Container Service: Container 1}

\gxs{\href{https://www.google.com/search?q=AWS+Elastic+Container+Service+Container+2}{Elastic Container Service: Container 2} \index{Elastic Container Service!Container 2}\index{Container 2 (Elastic Container Service)}} {\RECSContainerTwo{\iconsize}} {Res\_Amazon-Elastic-Container-Service\_Container-2\_48\_Light.pdf} {{\textbackslash}RECSContainerTwo} {Elastic Container Service: Container 2}

\gxs{\href{https://www.google.com/search?q=AWS+Elastic+Container+Service+Container+3}{Elastic Container Service: Container 3} \index{Elastic Container Service!Container 3}\index{Container 3 (Elastic Container Service)}} {\RECSContainerThree{\iconsize}} {Res\_Amazon-Elastic-Container-Service\_Container-3\_48\_Light.pdf} {{\textbackslash}RECSContainerThree} {Elastic Container Service: Container 3}

\gxs{\href{https://www.google.com/search?q=AWS+Elastic+Container+Service+Service}{Elastic Container Service: Service} \index{Elastic Container Service!Service}\index{Service (Elastic Container Service)}} {\RECSSvc{\iconsize}} {Res\_Amazon-Elastic-Container-Service\_Service\_48\_Light.pdf} {{\textbackslash}RECSSvc} {Elastic Container Service: Service}

\gxs{\href{https://www.google.com/search?q=AWS+Elastic+Container+Service+Task}{Elastic Container Service: Task} \index{Elastic Container Service!Task}\index{Task (Elastic Container Service)}} {\RECSTask{\iconsize}} {Res\_Amazon-Elastic-Container-Service\_Task\_48\_Light.pdf} {{\textbackslash}RECSTask} {Elastic Container Service: Task}

\gxs{\href{https://www.google.com/search?q=AWS+Elastic+File+System+File+System}{Elastic File System: File System} \index{Elastic File System!File System}\index{File System (Elastic File System)}} {\REFSFileSystem{\iconsize}} {Res\_Amazon-Elastic-File-System\_File-System\_48\_Light.pdf} {{\textbackslash}REFSFileSystem} {Elastic File System: File System}

\gxs{\href{https://www.google.com/search?q=AWS+Elastic+Load+Balancing+Application+Load+Balancer}{Elastic Load Balancing: Application Load Balancer} \index{Elastic Load Balancing!Application Load Balancer}\index{Application Load Balancer (Elastic Load Balancing)}} {\RELBAppLB{\iconsize}} {Res\_Elastic-Load-Balancing\_Application-Load-Balancer\_48\_Light.pdf} {{\textbackslash}RELBAppLB} {Elastic Load Balancing: Application Load Balancer}

\gxs{\href{https://www.google.com/search?q=AWS+Elastic+Load+Balancing+Classic+Load+Balancer}{Elastic Load Balancing: Classic Load Balancer} \index{Elastic Load Balancing!Classic Load Balancer}\index{Classic Load Balancer (Elastic Load Balancing)}} {\RELBClassicLB{\iconsize}} {Res\_Elastic-Load-Balancing\_Classic-Load-Balancer\_48\_Light.pdf} {{\textbackslash}RELBClassicLB} {Elastic Load Balancing: Classic Load Balancer}

\gxs{\href{https://www.google.com/search?q=AWS+Elastic+Load+Balancing+Network+Load+Balancer}{Elastic Load Balancing: Network Load Balancer} \index{Elastic Load Balancing!Network Load Balancer}\index{Network Load Balancer (Elastic Load Balancing)}} {\RELBNetworkLB{\iconsize}} {Res\_Elastic-Load-Balancing\_Network-Load-Balancer\_48\_Light.pdf} {{\textbackslash}RELBNetworkLB} {Elastic Load Balancing: Network Load Balancer}

\gxs{\href{https://www.google.com/search?q=AWS+Email}{Email} \index{Email}} {\REmail{\iconsize}} {Res\_Email\_48\_Light.pdf} {{\textbackslash}REmail} {Email}

\gxs{\href{https://www.google.com/search?q=AWS+EventBridge+Event}{EventBridge Event} \index{EventBridge Event}} {\REvBrEvent{\iconsize}} {Res\_Amazon-EventBridge-Event\_48\_Light.pdf} {{\textbackslash}REvBrEvent} {EventBridge Event}

\gxs{\href{https://www.google.com/search?q=AWS+EventBridge+Rule}{EventBridge Rule} \index{EventBridge Rule}} {\REvBrRule{\iconsize}} {Res\_Amazon-EventBridge-Rule\_48\_Light.pdf} {{\textbackslash}REvBrRule} {EventBridge Rule}

\gxs{\href{https://www.google.com/search?q=AWS+EventBridge+Saas+Partner+Event}{EventBridge Saas Partner Event} \index{EventBridge Saas Partner Event}} {\REvBrSaasPartnerEvent{\iconsize}} {Res\_Amazon-EventBridge-Saas-Partner-Event\_48\_Light.pdf} {{\textbackslash}REvBrSaasPartnerEvent} {EventBridge Saas Partner Event}

\gxs{\href{https://www.google.com/search?q=AWS+EventBridge+Custom+Event+Bus}{EventBridge: Custom Event Bus} \index{EventBridge!Custom Event Bus}\index{Custom Event Bus (EventBridge)}} {\REvBrCustomEventBus{\iconsize}} {Res\_Amazon-EventBridge\_Custom-Event-Bus\_48\_Light.pdf} {{\textbackslash}REvBrCustomEventBus} {EventBridge: Custom Event Bus}

\gxs{\href{https://www.google.com/search?q=AWS+EventBridge+Default+Event+Bus}{EventBridge: Default Event Bus} \index{EventBridge!Default Event Bus}\index{Default Event Bus (EventBridge)}} {\REvBrDfltEventBus{\iconsize}} {Res\_Amazon-EventBridge\_Default-Event-Bus\_48\_Light.pdf} {{\textbackslash}REvBrDfltEventBus} {EventBridge: Default Event Bus}

\gxs{\href{https://www.google.com/search?q=AWS+Firewall}{Firewall} \index{Firewall}} {\RFirewall{\iconsize}} {Res\_Firewall\_48\_Light.pdf} {{\textbackslash}RFirewall} {Firewall}

\gxs{\href{https://www.google.com/search?q=AWS+Glue+Crawler}{Glue: Crawler} \index{Glue!Crawler}\index{Crawler (Glue)}} {\RGlueCrawler{\iconsize}} {Res\_AWS-Glue\_Crawler\_48\_Light.pdf} {{\textbackslash}RGlueCrawler} {Glue: Crawler}

\gxs{\href{https://www.google.com/search?q=AWS+Glue+Data+Catalog}{Glue: Data Catalog} \index{Glue!Data Catalog}\index{Data Catalog (Glue)}} {\RGlueDataCatalog{\iconsize}} {Res\_AWS-Glue\_Data-Catalog\_48\_Light.pdf} {{\textbackslash}RGlueDataCatalog} {Glue: Data Catalog}

\gxs{\href{https://www.google.com/search?q=AWS+Identity+Access+Management+AWS+IAM+Access+Analyzer}{Identity Access Management: AWS IAM Access Analyzer} \index{Identity Access Management!AWS IAM Access Analyzer}\index{AWS IAM Access Analyzer (Identity Access Management)}} {\RIAMIAMAccessAnalyzer{\iconsize}} {Res\_AWS-Identity-Access-Management\_AWS-IAM-Access-Analyzer\_48\_Light.pdf} {{\textbackslash}RIAMIAMAccessAnalyzer} {Identity Access Management: AWS IAM Access Analyzer}

\gxs{\href{https://www.google.com/search?q=AWS+Identity+Access+Management+AWS+STS}{Identity Access Management: AWS STS} \index{Identity Access Management!AWS STS}\index{AWS STS (Identity Access Management)}} {\RIAMSTS{\iconsize}} {Res\_AWS-Identity-Access-Management\_AWS-STS\_48\_Light.pdf} {{\textbackslash}RIAMSTS} {Identity Access Management: AWS STS}

\gxs{\href{https://www.google.com/search?q=AWS+Identity+Access+Management+AWS+STS+Alternate}{Identity Access Management: AWS STS Alternate} \index{Identity Access Management!AWS STS Alternate}\index{AWS STS Alternate (Identity Access Management)}} {\RIAMSTSAltern{\iconsize}} {Res\_AWS-Identity-Access-Management\_AWS-STS-Alternate\_48\_Light.pdf} {{\textbackslash}RIAMSTSAltern} {Identity Access Management: AWS STS Alternate}

\gxs{\href{https://www.google.com/search?q=AWS+Identity+Access+Management+Add+on}{Identity Access Management: Add on} \index{Identity Access Management!Add on}\index{Add on (Identity Access Management)}} {\RIAMAddon{\iconsize}} {Res\_AWS-Identity-Access-Management\_Add-on\_48\_Light.pdf} {{\textbackslash}RIAMAddon} {Identity Access Management: Add on}

\gxs{\href{https://www.google.com/search?q=AWS+Identity+Access+Management+Data+Encryption+Key}{Identity Access Management: Data Encryption Key} \index{Identity Access Management!Data Encryption Key}\index{Data Encryption Key (Identity Access Management)}} {\RIAMDataEncrKey{\iconsize}} {Res\_AWS-Identity-Access-Management\_Data-Encryption-Key\_48\_Light.pdf} {{\textbackslash}RIAMDataEncrKey} {Identity Access Management: Data Encryption Key}

\gxs{\href{https://www.google.com/search?q=AWS+Identity+Access+Management+Encrypted+Data}{Identity Access Management: Encrypted Data} \index{Identity Access Management!Encrypted Data}\index{Encrypted Data (Identity Access Management)}} {\RIAMEncrdData{\iconsize}} {Res\_AWS-Identity-Access-Management\_Encrypted-Data\_48\_Light.pdf} {{\textbackslash}RIAMEncrdData} {Identity Access Management: Encrypted Data}

\gxs{\href{https://www.google.com/search?q=AWS+Identity+Access+Management+Long+Term+Security+Credential}{Identity Access Management: Long Term Security Credential} \index{Identity Access Management!Long Term Security Credential}\index{Long Term Security Credential (Identity Access Management)}} {\RIAMLongTermSecCred{\iconsize}} {Res\_AWS-Identity-Access-Management\_Long-Term-Security-Credential\_48\_Light.pdf} {{\textbackslash}RIAMLongTermSecCred} {Identity Access Management: Long Term Security Credential}

\gxs{\href{https://www.google.com/search?q=AWS+Identity+Access+Management+MFA+Token}{Identity Access Management: MFA Token} \index{Identity Access Management!MFA Token}\index{MFA Token (Identity Access Management)}} {\RIAMMFAToken{\iconsize}} {Res\_AWS-Identity-Access-Management\_MFA-Token\_48\_Light.pdf} {{\textbackslash}RIAMMFAToken} {Identity Access Management: MFA Token}

\gxs{\href{https://www.google.com/search?q=AWS+Identity+Access+Management+Permissions}{Identity Access Management: Permissions} \index{Identity Access Management!Permissions}\index{Permissions (Identity Access Management)}} {\RIAMPermissions{\iconsize}} {Res\_AWS-Identity-Access-Management\_Permissions\_48\_Light.pdf} {{\textbackslash}RIAMPermissions} {Identity Access Management: Permissions}

\gxs{\href{https://www.google.com/search?q=AWS+Identity+Access+Management+Role}{Identity Access Management: Role} \index{Identity Access Management!Role}\index{Role (Identity Access Management)}} {\RIAMRole{\iconsize}} {Res\_AWS-Identity-Access-Management\_Role\_48\_Light.pdf} {{\textbackslash}RIAMRole} {Identity Access Management: Role}

\gxs{\href{https://www.google.com/search?q=AWS+Identity+Access+Management+Temporary+Security+Credential}{Identity Access Management: Temporary Security Credential} \index{Identity Access Management!Temporary Security Credential}\index{Temporary Security Credential (Identity Access Management)}} {\RIAMTemporarySecCred{\iconsize}} {Res\_AWS-Identity-Access-Management\_Temporary-Security-Credential\_48\_Light.pdf} {{\textbackslash}RIAMTemporarySecCred} {Identity Access Management: Temporary Security Credential}

\gxs{\href{https://www.google.com/search?q=AWS+Inspector+Agent}{Inspector: Agent} \index{Inspector!Agent}\index{Agent (Inspector)}} {\RInspectorAgent{\iconsize}} {Res\_Amazon-Inspector\_Agent\_48\_Light.pdf} {{\textbackslash}RInspectorAgent} {Inspector: Agent}

\gxs{\href{https://www.google.com/search?q=AWS+IoT+Analytics+Channel}{IoT Analytics: Channel} \index{IoT Analytics!Channel}\index{Channel (IoT Analytics)}} {\RIoTAnalyticsChannel{\iconsize}} {Res\_AWS-IoT-Analytics\_Channel\_48\_Light.pdf} {{\textbackslash}RIoTAnalyticsChannel} {IoT Analytics: Channel}

\gxs{\href{https://www.google.com/search?q=AWS+IoT+Analytics+Data+Set}{IoT Analytics: Data Set} \index{IoT Analytics!Data Set}\index{Data Set (IoT Analytics)}} {\RIoTAnalyticsDataSet{\iconsize}} {Res\_AWS-IoT-Analytics\_Data-Set\_48\_Light.pdf} {{\textbackslash}RIoTAnalyticsDataSet} {IoT Analytics: Data Set}

\gxs{\href{https://www.google.com/search?q=AWS+IoT+Analytics+Data+Store}{IoT Analytics: Data Store} \index{IoT Analytics!Data Store}\index{Data Store (IoT Analytics)}} {\RIoTAnalyticsDataStore{\iconsize}} {Res\_AWS-IoT-Analytics\_Data-Store\_48\_Light.pdf} {{\textbackslash}RIoTAnalyticsDataStore} {IoT Analytics: Data Store}

\gxs{\href{https://www.google.com/search?q=AWS+IoT+Analytics+Notebook}{IoT Analytics: Notebook} \index{IoT Analytics!Notebook}\index{Notebook (IoT Analytics)}} {\RIoTAnalyticsNotebook{\iconsize}} {Res\_AWS-IoT-Analytics\_Notebook\_48\_Light.pdf} {{\textbackslash}RIoTAnalyticsNotebook} {IoT Analytics: Notebook}

\gxs{\href{https://www.google.com/search?q=AWS+IoT+Analytics+Pipeline}{IoT Analytics: Pipeline} \index{IoT Analytics!Pipeline}\index{Pipeline (IoT Analytics)}} {\RIoTAnalyticsPipeline{\iconsize}} {Res\_AWS-IoT-Analytics\_Pipeline\_48\_Light.pdf} {{\textbackslash}RIoTAnalyticsPipeline} {IoT Analytics: Pipeline}

\gxs{\href{https://www.google.com/search?q=AWS+IoT+Device+Defender+IoT+Device+Jobs}{IoT Device Defender: IoT Device Jobs} \index{IoT Device Defender!IoT Device Jobs}\index{IoT Device Jobs (IoT Device Defender)}} {\RIoTDevDefenderIoTDevJobs{\iconsize}} {Res\_AWS-loT-Device-Defender\_IoT-Device-Jobs\_48\_Light.pdf} {{\textbackslash}RIoTDevDefenderIoTDevJobs} {IoT Device Defender: IoT Device Jobs}

\gxs{\href{https://www.google.com/search?q=AWS+IoT+Greengrass+Connector}{IoT Greengrass: Connector} \index{IoT Greengrass!Connector}\index{Connector (IoT Greengrass)}} {\RIoTGreengrassConnector{\iconsize}} {Res\_AWS-IoT-Greengrass\_Connector\_48\_Light.pdf} {{\textbackslash}RIoTGreengrassConnector} {IoT Greengrass: Connector}

\gxs{\href{https://www.google.com/search?q=AWS+IoT+HTTP+Protocol}{IoT HTTP: Protocol} \index{IoT HTTP!Protocol}\index{Protocol (IoT HTTP)}} {\RIoTHTTPProtocol{\iconsize}} {Res\_IoT-HTTP\_Protocol\_48\_Light.pdf} {{\textbackslash}RIoTHTTPProtocol} {IoT HTTP: Protocol}

\gxs{\href{https://www.google.com/search?q=AWS+IoT+Hardware+Board}{IoT Hardware Board} \index{IoT Hardware Board}} {\RIoTHardwareBoard{\iconsize}} {Res\_IoT-Hardware-Board\_48\_Light.pdf} {{\textbackslash}RIoTHardwareBoard} {IoT Hardware Board}

\gxs{\href{https://www.google.com/search?q=AWS+IoT+MQTT+Protocol}{IoT MQTT: Protocol} \index{IoT MQTT!Protocol}\index{Protocol (IoT MQTT)}} {\RIoTMQTTProtocol{\iconsize}} {Res\_IoT-MQTT\_Protocol\_48\_Light.pdf} {{\textbackslash}RIoTMQTTProtocol} {IoT MQTT: Protocol}

\gxs{\href{https://www.google.com/search?q=AWS+IoT+Rule}{IoT Rule} \index{IoT Rule}} {\RIoTRule{\iconsize}} {Res\_IoT-Rule\_48\_Light.pdf} {{\textbackslash}RIoTRule} {IoT Rule}

\gxs{\href{https://www.google.com/search?q=AWS+IoT+Thing+Bank}{IoT Thing: Bank} \index{IoT Thing!Bank}\index{Bank (IoT Thing)}} {\RIoTThngBank{\iconsize}} {Res\_IoT-Thing\_Bank\_48\_Light.pdf} {{\textbackslash}RIoTThngBank} {IoT Thing: Bank}

\gxs{\href{https://www.google.com/search?q=AWS+IoT+Thing+Bicycle}{IoT Thing: Bicycle} \index{IoT Thing!Bicycle}\index{Bicycle (IoT Thing)}} {\RIoTThngBicycle{\iconsize}} {Res\_IoT-Thing\_Bicycle\_48\_Light.pdf} {{\textbackslash}RIoTThngBicycle} {IoT Thing: Bicycle}

\gxs{\href{https://www.google.com/search?q=AWS+IoT+Thing+Camera}{IoT Thing: Camera} \index{IoT Thing!Camera}\index{Camera (IoT Thing)}} {\RIoTThngCamera{\iconsize}} {Res\_IoT-Thing\_Camera\_48\_Light.pdf} {{\textbackslash}RIoTThngCamera} {IoT Thing: Camera}

\gxs{\href{https://www.google.com/search?q=AWS+IoT+Thing+Car}{IoT Thing: Car} \index{IoT Thing!Car}\index{Car (IoT Thing)}} {\RIoTThngCar{\iconsize}} {Res\_IoT-Thing\_Car\_48\_Light.pdf} {{\textbackslash}RIoTThngCar} {IoT Thing: Car}

\gxs{\href{https://www.google.com/search?q=AWS+IoT+Thing+Cart}{IoT Thing: Cart} \index{IoT Thing!Cart}\index{Cart (IoT Thing)}} {\RIoTThngCart{\iconsize}} {Res\_IoT-Thing\_Cart\_48\_Light.pdf} {{\textbackslash}RIoTThngCart} {IoT Thing: Cart}

\gxs{\href{https://www.google.com/search?q=AWS+IoT+Thing+Coffee+Pot}{IoT Thing: Coffee Pot} \index{IoT Thing!Coffee Pot}\index{Coffee Pot (IoT Thing)}} {\RIoTThngCoffeePot{\iconsize}} {Res\_IoT-Thing\_Coffee-Pot\_48\_Light.pdf} {{\textbackslash}RIoTThngCoffeePot} {IoT Thing: Coffee Pot}

\gxs{\href{https://www.google.com/search?q=AWS+IoT+Thing+Door+Lock}{IoT Thing: Door Lock} \index{IoT Thing!Door Lock}\index{Door Lock (IoT Thing)}} {\RIoTThngDoorLock{\iconsize}} {Res\_IoT-Thing\_Door-Lock\_48\_Light.pdf} {{\textbackslash}RIoTThngDoorLock} {IoT Thing: Door Lock}

\gxs{\href{https://www.google.com/search?q=AWS+IoT+Thing+Factory}{IoT Thing: Factory} \index{IoT Thing!Factory}\index{Factory (IoT Thing)}} {\RIoTThngFactory{\iconsize}} {Res\_IoT-Thing\_Factory\_48\_Light.pdf} {{\textbackslash}RIoTThngFactory} {IoT Thing: Factory}

\gxs{\href{https://www.google.com/search?q=AWS+IoT+Thing+Generic}{IoT Thing: Generic} \index{IoT Thing!Generic}\index{Generic (IoT Thing)}} {\RIoTThngGeneric{\iconsize}} {Res\_IoT-Thing\_Generic\_48\_Light.pdf} {{\textbackslash}RIoTThngGeneric} {IoT Thing: Generic}

\gxs{\href{https://www.google.com/search?q=AWS+IoT+Thing+House}{IoT Thing: House} \index{IoT Thing!House}\index{House (IoT Thing)}} {\RIoTThngHouse{\iconsize}} {Res\_IoT-Thing\_House\_48\_Light.pdf} {{\textbackslash}RIoTThngHouse} {IoT Thing: House}

\gxs{\href{https://www.google.com/search?q=AWS+IoT+Thing+Lightbulb}{IoT Thing: Lightbulb} \index{IoT Thing!Lightbulb}\index{Lightbulb (IoT Thing)}} {\RIoTThngLightbulb{\iconsize}} {Res\_IoT-Thing\_Lightbulb\_48\_Light.pdf} {{\textbackslash}RIoTThngLightbulb} {IoT Thing: Lightbulb}

\gxs{\href{https://www.google.com/search?q=AWS+IoT+Thing+Medical+Emergency}{IoT Thing: Medical Emergency} \index{IoT Thing!Medical Emergency}\index{Medical Emergency (IoT Thing)}} {\RIoTThngMedEmerg{\iconsize}} {Res\_IoT-Thing\_Medical-Emergency\_48\_Light.pdf} {{\textbackslash}RIoTThngMedEmerg} {IoT Thing: Medical Emergency}

\gxs{\href{https://www.google.com/search?q=AWS+IoT+Thing+Police+Emergency}{IoT Thing: Police Emergency} \index{IoT Thing!Police Emergency}\index{Police Emergency (IoT Thing)}} {\RIoTThngPoliceEmerg{\iconsize}} {Res\_IoT-Thing\_Police-Emergency\_48\_Light.pdf} {{\textbackslash}RIoTThngPoliceEmerg} {IoT Thing: Police Emergency}

\gxs{\href{https://www.google.com/search?q=AWS+IoT+Thing+Thermostat}{IoT Thing: Thermostat} \index{IoT Thing!Thermostat}\index{Thermostat (IoT Thing)}} {\RIoTThngThermostat{\iconsize}} {Res\_IoT-Thing\_Thermostat\_48\_Light.pdf} {{\textbackslash}RIoTThngThermostat} {IoT Thing: Thermostat}

\gxs{\href{https://www.google.com/search?q=AWS+IoT+Thing+Travel}{IoT Thing: Travel} \index{IoT Thing!Travel}\index{Travel (IoT Thing)}} {\RIoTThngTravel{\iconsize}} {Res\_IoT-Thing\_Travel\_48\_Light.pdf} {{\textbackslash}RIoTThngTravel} {IoT Thing: Travel}

\gxs{\href{https://www.google.com/search?q=AWS+IoT+Thing+Utility}{IoT Thing: Utility} \index{IoT Thing!Utility}\index{Utility (IoT Thing)}} {\RIoTThngUtility{\iconsize}} {Res\_IoT-Thing\_Utility\_48\_Light.pdf} {{\textbackslash}RIoTThngUtility} {IoT Thing: Utility}

\gxs{\href{https://www.google.com/search?q=AWS+IoT+Thing+Windfarm}{IoT Thing: Windfarm} \index{IoT Thing!Windfarm}\index{Windfarm (IoT Thing)}} {\RIoTThngWindfarm{\iconsize}} {Res\_IoT-Thing\_Windfarm\_48\_Light.pdf} {{\textbackslash}RIoTThngWindfarm} {IoT Thing: Windfarm}

\gxs{\href{https://www.google.com/search?q=AWS+IoT+Action}{IoT: Action} \index{IoT!Action}\index{Action (IoT)}} {\RIoTAction{\iconsize}} {Res\_IoT\_Action\_48\_Light.pdf} {{\textbackslash}RIoTAction} {IoT: Action}

\gxs{\href{https://www.google.com/search?q=AWS+IoT+Actuator}{IoT: Actuator} \index{IoT!Actuator}\index{Actuator (IoT)}} {\RIoTActuator{\iconsize}} {Res\_IoT\_Actuator\_48\_Light.pdf} {{\textbackslash}RIoTActuator} {IoT: Actuator}

\gxs{\href{https://www.google.com/search?q=AWS+IoT+Alexa+Enabled+Device}{IoT: Alexa: Enabled Device} \index{IoT!Alexa!Enabled Device}\index{Enabled Device (IoT Alexa)}} {\RIoTAlexaEnabledDev{\iconsize}} {Res\_IoT\_Alexa\_Enabled-Device\_48\_Light.pdf} {{\textbackslash}RIoTAlexaEnabledDev} {IoT: Alexa: Enabled Device}

\gxs{\href{https://www.google.com/search?q=AWS+IoT+Alexa+Skill}{IoT: Alexa: Skill} \index{IoT!Alexa!Skill}\index{Skill (IoT Alexa)}} {\RIoTAlexaSkill{\iconsize}} {Res\_IoT\_Alexa\_Skill\_48\_Light.pdf} {{\textbackslash}RIoTAlexaSkill} {IoT: Alexa: Skill}

\gxs{\href{https://www.google.com/search?q=AWS+IoT+Alexa+Voice+Service}{IoT: Alexa: Voice Service} \index{IoT!Alexa!Voice Service}\index{Voice Service (IoT Alexa)}} {\RIoTAlexaVoiceSvc{\iconsize}} {Res\_IoT\_Alexa\_Voice-Service\_48\_Light.pdf} {{\textbackslash}RIoTAlexaVoiceSvc} {IoT: Alexa: Voice Service}

\gxs{\href{https://www.google.com/search?q=AWS+IoT+Certificate}{IoT: Certificate} \index{IoT!Certificate}\index{Certificate (IoT)}} {\RIoTCert{\iconsize}} {Res\_IoT\_Certificate\_48\_Light.pdf} {{\textbackslash}RIoTCert} {IoT: Certificate}

\gxs{\href{https://www.google.com/search?q=AWS+IoT+Desired+State}{IoT: Desired State} \index{IoT!Desired State}\index{Desired State (IoT)}} {\RIoTDesiredState{\iconsize}} {Res\_IoT\_Desired-State\_48\_Light.pdf} {{\textbackslash}RIoTDesiredState} {IoT: Desired State}

\gxs{\href{https://www.google.com/search?q=AWS+IoT+Device+Gateway}{IoT: Device Gateway} \index{IoT!Device Gateway}\index{Device Gateway (IoT)}} {\RIoTDevGateway{\iconsize}} {Res\_IoT\_Device-Gateway\_48\_Light.pdf} {{\textbackslash}RIoTDevGateway} {IoT: Device Gateway}

\gxs{\href{https://www.google.com/search?q=AWS+IoT+Echo}{IoT: Echo} \index{IoT!Echo}\index{Echo (IoT)}} {\RIoTEcho{\iconsize}} {Res\_IoT\_Echo\_48\_Light.pdf} {{\textbackslash}RIoTEcho} {IoT: Echo}

\gxs{\href{https://www.google.com/search?q=AWS+IoT+Fire+TV+Stick}{IoT: Fire TV: Stick} \index{IoT!Fire TV!Stick}\index{Stick (IoT Fire TV)}} {\RIoTFireTVStick{\iconsize}} {Res\_IoT\_Fire-TV\_Stick\_48\_Light.pdf} {{\textbackslash}RIoTFireTVStick} {IoT: Fire TV: Stick}

\gxs{\href{https://www.google.com/search?q=AWS+IoT+Fire+TV}{IoT: Fire: TV} \index{IoT!Fire!TV}\index{TV (IoT Fire)}} {\RIoTFireTV{\iconsize}} {Res\_IoT\_Fire\_TV\_48\_Light.pdf} {{\textbackslash}RIoTFireTV} {IoT: Fire: TV}

\gxs{\href{https://www.google.com/search?q=AWS+IoT+HTTP2+Protocol}{IoT: HTTP2 Protocol} \index{IoT!HTTP2 Protocol}\index{HTTP2 Protocol (IoT)}} {\RIoTHTTPTwoProtocol{\iconsize}} {Res\_IoT\_HTTP2-Protocol\_48\_Light.pdf} {{\textbackslash}RIoTHTTPTwoProtocol} {IoT: HTTP2 Protocol}

\gxs{\href{https://www.google.com/search?q=AWS+IoT+Lambda+Function}{IoT: Lambda: Function} \index{IoT!Lambda!Function}\index{Function (IoT Lambda)}} {\RIoTLambdaFunc{\iconsize}} {Res\_IoT\_Lambda\_Function\_48\_Light.pdf} {{\textbackslash}RIoTLambdaFunc} {IoT: Lambda: Function}

\gxs{\href{https://www.google.com/search?q=AWS+IoT+Over+Air+Update}{IoT: Over Air Update} \index{IoT!Over Air Update}\index{Over Air Update (IoT)}} {\RIoTOverAirUpdate{\iconsize}} {Res\_IoT\_Over-Air-Update\_48\_Light.pdf} {{\textbackslash}RIoTOverAirUpdate} {IoT: Over Air Update}

\gxs{\href{https://www.google.com/search?q=AWS+IoT+Policy}{IoT: Policy} \index{IoT!Policy}\index{Policy (IoT)}} {\RIoTPolicy{\iconsize}} {Res\_IoT\_Policy\_48\_Light.pdf} {{\textbackslash}RIoTPolicy} {IoT: Policy}

\gxs{\href{https://www.google.com/search?q=AWS+IoT+Reported+State}{IoT: Reported State} \index{IoT!Reported State}\index{Reported State (IoT)}} {\RIoTReportedState{\iconsize}} {Res\_IoT\_Reported-State\_48\_Light.pdf} {{\textbackslash}RIoTReportedState} {IoT: Reported State}

\gxs{\href{https://www.google.com/search?q=AWS+IoT+Sensor}{IoT: Sensor} \index{IoT!Sensor}\index{Sensor (IoT)}} {\RIoTSensor{\iconsize}} {Res\_IoT\_Sensor\_48\_Light.pdf} {{\textbackslash}RIoTSensor} {IoT: Sensor}

\gxs{\href{https://www.google.com/search?q=AWS+IoT+Servo}{IoT: Servo} \index{IoT!Servo}\index{Servo (IoT)}} {\RIoTServo{\iconsize}} {Res\_IoT\_Servo\_48\_Light.pdf} {{\textbackslash}RIoTServo} {IoT: Servo}

\gxs{\href{https://www.google.com/search?q=AWS+IoT+Shadow}{IoT: Shadow} \index{IoT!Shadow}\index{Shadow (IoT)}} {\RIoTShadow{\iconsize}} {Res\_IoT\_Shadow\_48\_Light.pdf} {{\textbackslash}RIoTShadow} {IoT: Shadow}

\gxs{\href{https://www.google.com/search?q=AWS+IoT+Simulator}{IoT: Simulator} \index{IoT!Simulator}\index{Simulator (IoT)}} {\RIoTSimulator{\iconsize}} {Res\_IoT\_Simulator\_48\_Light.pdf} {{\textbackslash}RIoTSimulator} {IoT: Simulator}

\gxs{\href{https://www.google.com/search?q=AWS+IoT+Topic}{IoT: Topic} \index{IoT!Topic}\index{Topic (IoT)}} {\RIoTTopic{\iconsize}} {Res\_IoT\_Topic\_48\_Light.pdf} {{\textbackslash}RIoTTopic} {IoT: Topic}

\gxs{\href{https://www.google.com/search?q=AWS+Lake+Formation+Data+Lake}{Lake Formation: Data Lake} \index{Lake Formation!Data Lake}\index{Data Lake (Lake Formation)}} {\RLakeFormationDataLake{\iconsize}} {Res\_AWS-Lake-Formation\_Data-Lake\_48\_Light.pdf} {{\textbackslash}RLakeFormationDataLake} {Lake Formation: Data Lake}

\gxs{\href{https://www.google.com/search?q=AWS+Lambda+Lambda+Function}{Lambda: Lambda Function} \index{Lambda!Lambda Function}\index{Lambda Function (Lambda)}} {\RLambdaLambdaFunc{\iconsize}} {Res\_Amazon-Lambda\_Lambda-Function\_48\_Light.pdf} {{\textbackslash}RLambdaLambdaFunc} {Lambda: Lambda Function}

\gxs{\href{https://www.google.com/search?q=AWS+Managed+Blockchain+Blockchain}{Managed Blockchain: Blockchain} \index{Managed Blockchain!Blockchain}\index{Blockchain (Managed Blockchain)}} {\RMngdBlockchainBlockchain{\iconsize}} {Res\_Amazon-Managed-Blockchain\_Blockchain\_48\_Light.pdf} {{\textbackslash}RMngdBlockchainBlockchain} {Managed Blockchain: Blockchain}

\gxs{\href{https://www.google.com/search?q=AWS+OpsWorks+Apps}{OpsWorks: Apps} \index{OpsWorks!Apps}\index{Apps (OpsWorks)}} {\ROpWkApps{\iconsize}} {Res\_AWS-OpsWorks\_Apps\_48\_Light.pdf} {{\textbackslash}ROpWkApps} {OpsWorks: Apps}

\gxs{\href{https://www.google.com/search?q=AWS+OpsWorks+Deployments}{OpsWorks: Deployments} \index{OpsWorks!Deployments}\index{Deployments (OpsWorks)}} {\ROpWkDeployments{\iconsize}} {Res\_AWS-OpsWorks\_Deployments\_48\_Light.pdf} {{\textbackslash}ROpWkDeployments} {OpsWorks: Deployments}

\gxs{\href{https://www.google.com/search?q=AWS+OpsWorks+Instances}{OpsWorks: Instances} \index{OpsWorks!Instances}\index{Instances (OpsWorks)}} {\ROpWkInsts{\iconsize}} {Res\_AWS-OpsWorks\_Instances\_48\_Light.pdf} {{\textbackslash}ROpWkInsts} {OpsWorks: Instances}

\gxs{\href{https://www.google.com/search?q=AWS+OpsWorks+Layers}{OpsWorks: Layers} \index{OpsWorks!Layers}\index{Layers (OpsWorks)}} {\ROpWkLayers{\iconsize}} {Res\_AWS-OpsWorks\_Layers\_48\_Light.pdf} {{\textbackslash}ROpWkLayers} {OpsWorks: Layers}

\gxs{\href{https://www.google.com/search?q=AWS+OpsWorks+Monitoring}{OpsWorks: Monitoring} \index{OpsWorks!Monitoring}\index{Monitoring (OpsWorks)}} {\ROpWkMonitoring{\iconsize}} {Res\_AWS-OpsWorks-Monitoring\_48\_Light.pdf} {{\textbackslash}ROpWkMonitoring} {OpsWorks: Monitoring}

\gxs{\href{https://www.google.com/search?q=AWS+OpsWorks+Permissions}{OpsWorks: Permissions} \index{OpsWorks!Permissions}\index{Permissions (OpsWorks)}} {\ROpWkPermissions{\iconsize}} {Res\_AWS-OpsWorks\_Permissions\_48\_Light.pdf} {{\textbackslash}ROpWkPermissions} {OpsWorks: Permissions}

\gxs{\href{https://www.google.com/search?q=AWS+OpsWorks+Resources}{OpsWorks: Resources} \index{OpsWorks!Resources}\index{Resources (OpsWorks)}} {\ROpWkRess{\iconsize}} {Res\_AWS-OpsWorks\_Resources\_48\_Light.pdf} {{\textbackslash}ROpWkRess} {OpsWorks: Resources}

\gxs{\href{https://www.google.com/search?q=AWS+OpsWorks+Stack2}{OpsWorks: Stack2} \index{OpsWorks!Stack2}\index{Stack2 (OpsWorks)}} {\ROpWkStackTwo{\iconsize}} {Res\_AWS-OpsWorks-Stack2\_48\_Light.pdf} {{\textbackslash}ROpWkStackTwo} {OpsWorks: Stack2}

\gxs{\href{https://www.google.com/search?q=AWS+Organizations+Account}{Organizations: Account} \index{Organizations!Account}\index{Account (Organizations)}} {\ROrganizationsAccount{\iconsize}} {Res\_AWS-Organizations\_Account\_48\_Light.pdf} {{\textbackslash}ROrganizationsAccount} {Organizations: Account}

\gxs{\href{https://www.google.com/search?q=AWS+Organizations+Organizational+Unit}{Organizations: Organizational Unit} \index{Organizations!Organizational Unit}\index{Organizational Unit (Organizations)}} {\ROrganizationsOrgUnit{\iconsize}} {Res\_AWS-Organizations\_Organizational-Unit\_48\_Light.pdf} {{\textbackslash}ROrganizationsOrgUnit} {Organizations: Organizational Unit}

\gxs{\href{https://www.google.com/search?q=AWS+RDS+Proxy}{RDS Proxy} \index{RDS Proxy}} {\RRDSProxy{\iconsize}} {Res\_Amazon-RDS-Proxy\_48\_Light.pdf} {{\textbackslash}RRDSProxy} {RDS Proxy}

\gxs{\href{https://www.google.com/search?q=AWS+RDS+Proxy+Alternate}{RDS Proxy Alternate} \index{RDS Proxy Alternate}} {\RRDSProxyAltern{\iconsize}} {Res\_Amazon-RDS-Proxy-Alternate\_48\_Light.pdf} {{\textbackslash}RRDSProxyAltern} {RDS Proxy Alternate}

\gxs{\href{https://www.google.com/search?q=AWS+Redshift+Dense+Compute+Node}{Redshift: Dense Compute Node} \index{Redshift!Dense Compute Node}\index{Dense Compute Node (Redshift)}} {\RRedshiftDenseCmputNode{\iconsize}} {Res\_Amazon-Redshift\_Dense-Compute-Node\_48\_Light.pdf} {{\textbackslash}RRedshiftDenseCmputNode} {Redshift: Dense Compute Node}

\gxs{\href{https://www.google.com/search?q=AWS+Redshift+Dense+Storage+Node}{Redshift: Dense Storage Node} \index{Redshift!Dense Storage Node}\index{Dense Storage Node (Redshift)}} {\RRedshiftDenseStorageNode{\iconsize}} {Res\_Amazon-Redshift\_Dense-Storage-Node\_48\_Light.pdf} {{\textbackslash}RRedshiftDenseStorageNode} {Redshift: Dense Storage Node}

\gxs{\href{https://www.google.com/search?q=AWS+Rekognition+Image}{Rekognition: Image} \index{Rekognition!Image}\index{Image (Rekognition)}} {\RRekognitionImg{\iconsize}} {Res\_Amazon-Rekognition\_Image\_48\_Light.pdf} {{\textbackslash}RRekognitionImg} {Rekognition: Image}

\gxs{\href{https://www.google.com/search?q=AWS+Rekognition+Video}{Rekognition: Video} \index{Rekognition!Video}\index{Video (Rekognition)}} {\RRekognitionVideo{\iconsize}} {Res\_Amazon-Rekognition\_Video\_48\_Light.pdf} {{\textbackslash}RRekognitionVideo} {Rekognition: Video}

\gxs{\href{https://www.google.com/search?q=AWS+RoboMaker+Cloud+Extensions+ROS}{RoboMaker: Cloud Extensions ROS} \index{RoboMaker!Cloud Extensions ROS}\index{Cloud Extensions ROS (RoboMaker)}} {\RRoboMakerCloudExtensionsROS{\iconsize}} {Res\_AWS-RoboMaker\_Cloud-Extensions-ROS\_48\_Light.pdf} {{\textbackslash}RRoboMakerCloudExtensionsROS} {RoboMaker: Cloud Extensions ROS}

\gxs{\href{https://www.google.com/search?q=AWS+RoboMaker+Development+Environment}{RoboMaker: Development Environment} \index{RoboMaker!Development Environment}\index{Development Environment (RoboMaker)}} {\RRoboMakerDevEnvironment{\iconsize}} {Res\_AWS-RoboMaker\_Development-Environment\_48\_Light.pdf} {{\textbackslash}RRoboMakerDevEnvironment} {RoboMaker: Development Environment}

\gxs{\href{https://www.google.com/search?q=AWS+RoboMaker+Fleet+Management}{RoboMaker: Fleet Management} \index{RoboMaker!Fleet Management}\index{Fleet Management (RoboMaker)}} {\RRoboMakerFleetManagement{\iconsize}} {Res\_AWS-RoboMaker\_Fleet-Management\_48\_Light.pdf} {{\textbackslash}RRoboMakerFleetManagement} {RoboMaker: Fleet Management}

\gxs{\href{https://www.google.com/search?q=AWS+RoboMaker+Simulation}{RoboMaker: Simulation} \index{RoboMaker!Simulation}\index{Simulation (RoboMaker)}} {\RRoboMakerSimul{\iconsize}} {Res\_AWS-RoboMaker\_Simulation\_48\_Light.pdf} {{\textbackslash}RRoboMakerSimul} {RoboMaker: Simulation}

\gxs{\href{https://www.google.com/search?q=AWS+Route+53+Hosted+Zone}{Route 53 Hosted Zone} \index{Route 53 Hosted Zone}} {\RRouteFiveThreeHostedZone{\iconsize}} {Res\_Amazon-Route-53-Hosted-Zone\_48\_Light.pdf} {{\textbackslash}RRouteFiveThreeHostedZone} {Route 53 Hosted Zone}

\gxs{\href{https://www.google.com/search?q=AWS+Route+53+Route+Table}{Route 53: Route Table} \index{Route 53!Route Table}\index{Route Table (Route 53)}} {\RRouteFiveThreeRouteTable{\iconsize}} {Res\_Amazon-Route-53\_Route-Table\_48\_Light.pdf} {{\textbackslash}RRouteFiveThreeRouteTable} {Route 53: Route Table}

\gxs{\href{https://www.google.com/search?q=AWS+S3+Glacier+Archive}{S3 Glacier: Archive} \index{S3 Glacier!Archive}\index{Archive (S3 Glacier)}} {\RSThreeGlacierArchv{\iconsize}} {Res\_Amazon-S3-Glacier\_Archive\_48\_Light.pdf} {{\textbackslash}RSThreeGlacierArchv} {S3 Glacier: Archive}

\gxs{\href{https://www.google.com/search?q=AWS+S3+Glacier+Vault}{S3 Glacier: Vault} \index{S3 Glacier!Vault}\index{Vault (S3 Glacier)}} {\RSThreeGlacierVlt{\iconsize}} {Res\_Amazon-S3-Glacier\_Vault\_48\_Light.pdf} {{\textbackslash}RSThreeGlacierVlt} {S3 Glacier: Vault}

\gxs{\href{https://www.google.com/search?q=AWS+S3+Bucket}{S3: Bucket} \index{S3!Bucket}\index{Bucket (S3)}} {\RSThreeBkt{\iconsize}} {Res\_Amazon-Simple-Storage\_Bucket\_48\_Light.pdf} {{\textbackslash}RSThreeBkt} {S3: Bucket}

\gxs{\href{https://www.google.com/search?q=AWS+S3+Bucket+With+Objects}{S3: Bucket With Objects} \index{S3!Bucket With Objects}\index{Bucket With Objects (S3)}} {\RSThreeBktWithObjs{\iconsize}} {Res\_Amazon-Simple-Storage\_Bucket-With-Objects\_48\_Light.pdf} {{\textbackslash}RSThreeBktWithObjs} {S3: Bucket With Objects}

\gxs{\href{https://www.google.com/search?q=AWS+S3+General+Access+Points}{S3: General Access Points} \index{S3!General Access Points}\index{General Access Points (S3)}} {\RSThreeGenAccessPts{\iconsize}} {Res\_Amazon-Simple-Storage\_General-Access-Points\_48\_Light.pdf} {{\textbackslash}RSThreeGenAccessPts} {S3: General Access Points}

\gxs{\href{https://www.google.com/search?q=AWS+S3+Object}{S3: Object} \index{S3!Object}\index{Object (S3)}} {\RSThreeObj{\iconsize}} {Res\_Amazon-Simple-Storage\_Object\_48\_Light.pdf} {{\textbackslash}RSThreeObj} {S3: Object}

\gxs{\href{https://www.google.com/search?q=AWS+S3+S3+Replication}{S3: S3 Replication} \index{S3!S3 Replication}\index{S3 Replication (S3)}} {\RSThreeRepli{\iconsize}} {Res\_AWS-Amazon-Simple-Storage\_S3-Replication\_48\_Light.pdf} {{\textbackslash}RSThreeRepli} {S3: S3 Replication}

\gxs{\href{https://www.google.com/search?q=AWS+S3+S3+Replication+Time+Control}{S3: S3 Replication Time Control} \index{S3!S3 Replication Time Control}\index{S3 Replication Time Control (S3)}} {\RSThreeRepliTimeControl{\iconsize}} {Res\_AWS-Amazon-Simple-Storage\_S3-Replication-Time-Control\_48\_Light.pdf} {{\textbackslash}RSThreeRepliTimeControl} {S3: S3 Replication Time Control}

\gxs{\href{https://www.google.com/search?q=AWS+S3+Service+Glacier}{S3: Service Glacier} \index{S3!Service Glacier}\index{Service Glacier (S3)}} {\RSThreeSvcGlacier{\iconsize}} {Res\_Amazon-Simple-Storage\_Service-Glacier\_48\_Light.pdf} {{\textbackslash}RSThreeSvcGlacier} {S3: Service Glacier}

\gxs{\href{https://www.google.com/search?q=AWS+S3+Service+Glacier+Deep+Archive}{S3: Service Glacier Deep Archive} \index{S3!Service Glacier Deep Archive}\index{Service Glacier Deep Archive (S3)}} {\RSThreeSvcGlacierDeepArchv{\iconsize}} {Res\_Amazon-Simple-Storage\_Service-Glacier-Deep-Archive\_48\_Light.pdf} {{\textbackslash}RSThreeSvcGlacierDeepArchv} {S3: Service Glacier Deep Archive}

\gxs{\href{https://www.google.com/search?q=AWS+S3+Service+Intelligent+Tiering}{S3: Service Intelligent Tiering} \index{S3!Service Intelligent Tiering}\index{Service Intelligent Tiering (S3)}} {\RSThreeSvcIntellTiering{\iconsize}} {Res\_Amazon-Simple-Storage\_Service-Intelligent-Tiering\_48\_Light.pdf} {{\textbackslash}RSThreeSvcIntellTiering} {S3: Service Intelligent Tiering}

\gxs{\href{https://www.google.com/search?q=AWS+S3+Service+One+Zone+IA}{S3: Service One Zone IA} \index{S3!Service One Zone IA}\index{Service One Zone IA (S3)}} {\RSThreeSvcOneZoneIA{\iconsize}} {Res\_Amazon-Simple-Storage\_Service-One-Zone-IA\_48\_Light.pdf} {{\textbackslash}RSThreeSvcOneZoneIA} {S3: Service One Zone IA}

\gxs{\href{https://www.google.com/search?q=AWS+S3+Service+Standard}{S3: Service Standard} \index{S3!Service Standard}\index{Service Standard (S3)}} {\RSThreeSvcStandard{\iconsize}} {Res\_Amazon-Simple-Storage\_Service-Standard\_48\_Light.pdf} {{\textbackslash}RSThreeSvcStandard} {S3: Service Standard}

\gxs{\href{https://www.google.com/search?q=AWS+S3+Service+Standard+IA}{S3: Service Standard IA} \index{S3!Service Standard IA}\index{Service Standard IA (S3)}} {\RSThreeSvcStandardIA{\iconsize}} {Res\_Amazon-Simple-Storage\_Service-Standard-IA\_48\_Light.pdf} {{\textbackslash}RSThreeSvcStandardIA} {S3: Service Standard IA}

\gxs{\href{https://www.google.com/search?q=AWS+S3+VPC+Access+Points}{S3: VPC Access Points} \index{S3!VPC Access Points}\index{VPC Access Points (S3)}} {\RSThreeVPCAccessPts{\iconsize}} {Res\_Amazon-Simple-Storage\_VPC-Access-Points\_48\_Light.pdf} {{\textbackslash}RSThreeVPCAccessPts} {S3: VPC Access Points}

\gxs{\href{https://www.google.com/search?q=AWS+Sagemaker+Model}{Sagemaker: Model} \index{Sagemaker!Model}\index{Model (Sagemaker)}} {\RSagemakerModel{\iconsize}} {Res\_Amazon-Sagemaker\_Model\_48\_Light.pdf} {{\textbackslash}RSagemakerModel} {Sagemaker: Model}

\gxs{\href{https://www.google.com/search?q=AWS+Sagemaker+Notebook}{Sagemaker: Notebook} \index{Sagemaker!Notebook}\index{Notebook (Sagemaker)}} {\RSagemakerNotebook{\iconsize}} {Res\_Amazon-Sagemaker\_Notebook\_48\_Light.pdf} {{\textbackslash}RSagemakerNotebook} {Sagemaker: Notebook}

\gxs{\href{https://www.google.com/search?q=AWS+Sagemaker+Train}{Sagemaker: Train} \index{Sagemaker!Train}\index{Train (Sagemaker)}} {\RSagemakerTrain{\iconsize}} {Res\_Amazon-Sagemaker\_Train\_48\_Light.pdf} {{\textbackslash}RSagemakerTrain} {Sagemaker: Train}

\gxs{\href{https://www.google.com/search?q=AWS+Security+Hub+Finding}{Security Hub: Finding} \index{Security Hub!Finding}\index{Finding (Security Hub)}} {\RSecHubFinding{\iconsize}} {Res\_AWS-Security-Hub\_Finding\_48\_Light.pdf} {{\textbackslash}RSecHubFinding} {Security Hub: Finding}

\gxs{\href{https://www.google.com/search?q=AWS+Shield+AWS+Shield+Advanced}{Shield: AWS Shield Advanced} \index{Shield!AWS Shield Advanced}\index{AWS Shield Advanced (Shield)}} {\RShieldShieldAdvanced{\iconsize}} {Res\_AWS-Shield\_AWS-Shield-Advanced\_48\_Light.pdf} {{\textbackslash}RShieldShieldAdvanced} {Shield: AWS Shield Advanced}

\gxs{\href{https://www.google.com/search?q=AWS+Simple+Email+Service+Email}{Simple Email Service: Email} \index{Simple Email Service!Email}\index{Email (Simple Email Service)}} {\RSimpleEmailSvcEmail{\iconsize}} {Res\_Amazon-Simple-Email-Service\_Email\_48\_Light.pdf} {{\textbackslash}RSimpleEmailSvcEmail} {Simple Email Service: Email}

\gxs{\href{https://www.google.com/search?q=AWS+Simple+Notification+Service+Email+Notification}{Simple Notification Service: Email Notification} \index{Simple Notification Service!Email Notification}\index{Email Notification (Simple Notification Service)}} {\RSimpleNotifSvcEmailNotif{\iconsize}} {Res\_Amazon-Simple-Notification-Service\_Email-Notification\_48\_Light.pdf} {{\textbackslash}RSimpleNotifSvcEmailNotif} {Simple Notification Service: Email Notification}

\gxs{\href{https://www.google.com/search?q=AWS+Simple+Notification+Service+HTTP+Notification}{Simple Notification Service: HTTP Notification} \index{Simple Notification Service!HTTP Notification}\index{HTTP Notification (Simple Notification Service)}} {\RSimpleNotifSvcHTTPNotif{\iconsize}} {Res\_Amazon-Simple-Notification-Service\_HTTP-Notification\_48\_Light.pdf} {{\textbackslash}RSimpleNotifSvcHTTPNotif} {Simple Notification Service: HTTP Notification}

\gxs{\href{https://www.google.com/search?q=AWS+Simple+Notification+Service+Topic}{Simple Notification Service: Topic} \index{Simple Notification Service!Topic}\index{Topic (Simple Notification Service)}} {\RSimpleNotifSvcTopic{\iconsize}} {Res\_Amazon-Simple-Notification-Service\_Topic\_48\_Light.pdf} {{\textbackslash}RSimpleNotifSvcTopic} {Simple Notification Service: Topic}

\gxs{\href{https://www.google.com/search?q=AWS+Simple+Queue+Service+Message}{Simple Queue Service: Message} \index{Simple Queue Service!Message}\index{Message (Simple Queue Service)}} {\RSQSMessage{\iconsize}} {Res\_Amazon-Simple-Queue-Service\_Message\_48\_Light.pdf} {{\textbackslash}RSQSMessage} {Simple Queue Service: Message}

\gxs{\href{https://www.google.com/search?q=AWS+Simple+Queue+Service+Queue}{Simple Queue Service: Queue} \index{Simple Queue Service!Queue}\index{Queue (Simple Queue Service)}} {\RSQSQueue{\iconsize}} {Res\_Amazon-Simple-Queue-Service\_Queue\_48\_Light.pdf} {{\textbackslash}RSQSQueue} {Simple Queue Service: Queue}

\gxs{\href{https://www.google.com/search?q=AWS+Snowball+Snowball+Import+Export}{Snowball: Snowball Import Export} \index{Snowball!Snowball Import Export}\index{Snowball Import Export (Snowball)}} {\RSnowballImportExport{\iconsize}} {Res\_AWS-Snowball\_Snowball-Import-Export\_48\_Light.pdf} {{\textbackslash}RSnowballImportExport} {Snowball: Snowball Import Export}

\gxs{\href{https://www.google.com/search?q=AWS+Storage+Gateway+Cached+Volume}{Storage Gateway: Cached Volume} \index{Storage Gateway!Cached Volume}\index{Cached Volume (Storage Gateway)}} {\RStorGatCachedVol{\iconsize}} {Res\_AWS-Storage-Gateway\_Cached-Volume\_48\_Light.pdf} {{\textbackslash}RStorGatCachedVol} {Storage Gateway: Cached Volume}

\gxs{\href{https://www.google.com/search?q=AWS+Storage+Gateway+File+Gateway}{Storage Gateway: File Gateway} \index{Storage Gateway!File Gateway}\index{File Gateway (Storage Gateway)}} {\RStorGatFileGateway{\iconsize}} {Res\_AWS-Storage-Gateway\_File-Gateway\_48\_Light.pdf} {{\textbackslash}RStorGatFileGateway} {Storage Gateway: File Gateway}

\gxs{\href{https://www.google.com/search?q=AWS+Storage+Gateway+Non+Cached+Volume}{Storage Gateway: Non Cached Volume} \index{Storage Gateway!Non Cached Volume}\index{Non Cached Volume (Storage Gateway)}} {\RStorGatNonCachedVol{\iconsize}} {Res\_AWS-Storage-Gateway\_Non-Cached-Volume\_48\_Light.pdf} {{\textbackslash}RStorGatNonCachedVol} {Storage Gateway: Non Cached Volume}

\gxs{\href{https://www.google.com/search?q=AWS+Storage+Gateway+Tape+Gateway}{Storage Gateway: Tape Gateway} \index{Storage Gateway!Tape Gateway}\index{Tape Gateway (Storage Gateway)}} {\RStorGatTapeGateway{\iconsize}} {Res\_AWS-Storage-Gateway\_Tape-Gateway\_48\_Light.pdf} {{\textbackslash}RStorGatTapeGateway} {Storage Gateway: Tape Gateway}

\gxs{\href{https://www.google.com/search?q=AWS+Storage+Gateway+Virtual+Tape+Library}{Storage Gateway: Virtual Tape Library} \index{Storage Gateway!Virtual Tape Library}\index{Virtual Tape Library (Storage Gateway)}} {\RStorGatVirtTapeLibrary{\iconsize}} {Res\_AWS-Storage-Gateway\_Virtual-Tape-Library\_48\_Light.pdf} {{\textbackslash}RStorGatVirtTapeLibrary} {Storage Gateway: Virtual Tape Library}

\gxs{\href{https://www.google.com/search?q=AWS+Storage+Gateway+Volume+Gateway}{Storage Gateway: Volume Gateway} \index{Storage Gateway!Volume Gateway}\index{Volume Gateway (Storage Gateway)}} {\RStorGatVolGateway{\iconsize}} {Res\_AWS-Storage-Gateway\_Volume-Gateway\_48\_Light.pdf} {{\textbackslash}RStorGatVolGateway} {Storage Gateway: Volume Gateway}

\gxs{\href{https://www.google.com/search?q=AWS+System+Manager+Automation}{System Manager: Automation} \index{System Manager!Automation}\index{Automation (System Manager)}} {\RSystemMngrAutomation{\iconsize}} {Res\_AWS-System-Manager\_Automation\_48\_Light.pdf} {{\textbackslash}RSystemMngrAutomation} {System Manager: Automation}

\gxs{\href{https://www.google.com/search?q=AWS+System+Manager+Documents}{System Manager: Documents} \index{System Manager!Documents}\index{Documents (System Manager)}} {\RSystemMngrDocuments{\iconsize}} {Res\_AWS-System-Manager\_Documents\_48\_Light.pdf} {{\textbackslash}RSystemMngrDocuments} {System Manager: Documents}

\gxs{\href{https://www.google.com/search?q=AWS+System+Manager+Inventory}{System Manager: Inventory} \index{System Manager!Inventory}\index{Inventory (System Manager)}} {\RSystemMngrInventory{\iconsize}} {Res\_AWS-System-Manager\_Inventory\_48\_Light.pdf} {{\textbackslash}RSystemMngrInventory} {System Manager: Inventory}

\gxs{\href{https://www.google.com/search?q=AWS+System+Manager+Maintenance+Windows}{System Manager: Maintenance Windows} \index{System Manager!Maintenance Windows}\index{Maintenance Windows (System Manager)}} {\RSystemMngrMaintenanceWindows{\iconsize}} {Res\_AWS-System-Manager\_Maintenance-Windows\_48\_Light.pdf} {{\textbackslash}RSystemMngrMaintenanceWindows} {System Manager: Maintenance Windows}

\gxs{\href{https://www.google.com/search?q=AWS+System+Manager+OpsCenter}{System Manager: OpsCenter} \index{System Manager!OpsCenter}\index{OpsCenter (System Manager)}} {\RSystemMngrOpsCenter{\iconsize}} {Res\_AWS-System-Manager\_OpsCenter\_48\_Light.pdf} {{\textbackslash}RSystemMngrOpsCenter} {System Manager: OpsCenter}

\gxs{\href{https://www.google.com/search?q=AWS+System+Manager+Parameter+Store}{System Manager: Parameter Store} \index{System Manager!Parameter Store}\index{Parameter Store (System Manager)}} {\RSystemMngrParameterStore{\iconsize}} {Res\_AWS-System-Manager\_Parameter-Store\_48\_Light.pdf} {{\textbackslash}RSystemMngrParameterStore} {System Manager: Parameter Store}

\gxs{\href{https://www.google.com/search?q=AWS+System+Manager+Patch+Manager}{System Manager: Patch Manager} \index{System Manager!Patch Manager}\index{Patch Manager (System Manager)}} {\RSystemMngrPatchMngr{\iconsize}} {Res\_AWS-System-Manager\_Patch-Manager\_48\_Light.pdf} {{\textbackslash}RSystemMngrPatchMngr} {System Manager: Patch Manager}

\gxs{\href{https://www.google.com/search?q=AWS+System+Manager+Run+Command}{System Manager: Run Command} \index{System Manager!Run Command}\index{Run Command (System Manager)}} {\RSystemMngrRunCommand{\iconsize}} {Res\_AWS-System-Manager\_Run-Command\_48\_Light.pdf} {{\textbackslash}RSystemMngrRunCommand} {System Manager: Run Command}

\gxs{\href{https://www.google.com/search?q=AWS+System+Manager+State+Manager}{System Manager: State Manager} \index{System Manager!State Manager}\index{State Manager (System Manager)}} {\RSystemMngrStateMngr{\iconsize}} {Res\_AWS-System-Manager\_State-Manager\_48\_Light.pdf} {{\textbackslash}RSystemMngrStateMngr} {System Manager: State Manager}

\gxs{\href{https://www.google.com/search?q=AWS+Transfer+Family+AWS+FTP}{Transfer Family: AWS FTP} \index{Transfer Family!AWS FTP}\index{AWS FTP (Transfer Family)}} {\RTranFamFTP{\iconsize}} {Res\_AWS-Transfer-Family\_AWS-FTP\_48\_Light.pdf} {{\textbackslash}RTranFamFTP} {Transfer Family: AWS FTP}

\gxs{\href{https://www.google.com/search?q=AWS+Transfer+Family+AWS+FTPS}{Transfer Family: AWS FTPS} \index{Transfer Family!AWS FTPS}\index{AWS FTPS (Transfer Family)}} {\RTranFamFTPS{\iconsize}} {Res\_AWS-Transfer-Family\_AWS-FTPS\_48\_Light.pdf} {{\textbackslash}RTranFamFTPS} {Transfer Family: AWS FTPS}

\gxs{\href{https://www.google.com/search?q=AWS+Transfer+Family+AWS+SFTP}{Transfer Family: AWS SFTP} \index{Transfer Family!AWS SFTP}\index{AWS SFTP (Transfer Family)}} {\RTranFamSFTP{\iconsize}} {Res\_AWS-Transfer-Family\_AWS-SFTP\_48\_Light.pdf} {{\textbackslash}RTranFamSFTP} {Transfer Family: AWS SFTP}

\gxs{\href{https://www.google.com/search?q=AWS+Trusted+Advisor+Checklist}{Trusted Advisor: Checklist} \index{Trusted Advisor!Checklist}\index{Checklist (Trusted Advisor)}} {\RTrusTAdvChecklist{\iconsize}} {Res\_AWS-Trusted-Advisor\_Checklist\_48\_Light.pdf} {{\textbackslash}RTrusTAdvChecklist} {Trusted Advisor: Checklist}

\gxs{\href{https://www.google.com/search?q=AWS+Trusted+Advisor+Checklist+Cost}{Trusted Advisor: Checklist Cost} \index{Trusted Advisor!Checklist Cost}\index{Checklist Cost (Trusted Advisor)}} {\RTrusTAdvChecklistCost{\iconsize}} {Res\_AWS-Trusted-Advisor\_Checklist-Cost\_48\_Light.pdf} {{\textbackslash}RTrusTAdvChecklistCost} {Trusted Advisor: Checklist Cost}

\gxs{\href{https://www.google.com/search?q=AWS+Trusted+Advisor+Checklist+Fault+Tolerant}{Trusted Advisor: Checklist Fault Tolerant} \index{Trusted Advisor!Checklist Fault Tolerant}\index{Checklist Fault Tolerant (Trusted Advisor)}} {\RTrusTAdvChecklistFaultTolerant{\iconsize}} {Res\_AWS-Trusted-Advisor\_Checklist-Fault-Tolerant\_48\_Light.pdf} {{\textbackslash}RTrusTAdvChecklistFaultTolerant} {Trusted Advisor: Checklist Fault Tolerant}

\gxs{\href{https://www.google.com/search?q=AWS+Trusted+Advisor+Checklist+Performance}{Trusted Advisor: Checklist Performance} \index{Trusted Advisor!Checklist Performance}\index{Checklist Performance (Trusted Advisor)}} {\RTrusTAdvChecklistPerformance{\iconsize}} {Res\_AWS-Trusted-Advisor\_Checklist-Performance\_48\_Light.pdf} {{\textbackslash}RTrusTAdvChecklistPerformance} {Trusted Advisor: Checklist Performance}

\gxs{\href{https://www.google.com/search?q=AWS+Trusted+Advisor+Checklist+Security}{Trusted Advisor: Checklist Security} \index{Trusted Advisor!Checklist Security}\index{Checklist Security (Trusted Advisor)}} {\RTrusTAdvChecklistSec{\iconsize}} {Res\_AWS-Trusted-Advisor\_Checklist-Security\_48\_Light.pdf} {{\textbackslash}RTrusTAdvChecklistSec} {Trusted Advisor: Checklist Security}

\gxs{\href{https://www.google.com/search?q=AWS+VPC+Customer+Gateway}{VPC: Customer Gateway} \index{VPC!Customer Gateway}\index{Customer Gateway (VPC)}} {\RVPCCustomerGateway{\iconsize}} {Res\_Amazon-VPC\_Customer-Gateway\_48\_Light.pdf} {{\textbackslash}RVPCCustomerGateway} {VPC: Customer Gateway}

\gxs{\href{https://www.google.com/search?q=AWS+VPC+Elastic+Network+Adapter}{VPC: Elastic Network Adapter} \index{VPC!Elastic Network Adapter}\index{Elastic Network Adapter (VPC)}} {\RVPCENA{\iconsize}} {Res\_Amazon-VPC\_Elastic-Network-Adapter\_48\_Light.pdf} {{\textbackslash}RVPCENA} {VPC: Elastic Network Adapter}

\gxs{\href{https://www.google.com/search?q=AWS+VPC+Elastic+Network+Interface}{VPC: Elastic Network Interface} \index{VPC!Elastic Network Interface}\index{Elastic Network Interface (VPC)}} {\RVPCENI{\iconsize}} {Res\_Amazon-VPC\_Elastic-Network-Interface\_48\_Light.pdf} {{\textbackslash}RVPCENI} {VPC: Elastic Network Interface}

\gxs{\href{https://www.google.com/search?q=AWS+VPC+Endpoints}{VPC: Endpoints} \index{VPC!Endpoints}\index{Endpoints (VPC)}} {\RVPCEndpoints{\iconsize}} {Res\_Amazon-VPC\_Endpoints\_48\_Light.pdf} {{\textbackslash}RVPCEndpoints} {VPC: Endpoints}

\gxs{\href{https://www.google.com/search?q=AWS+VPC+Flow+Logs}{VPC: Flow Logs} \index{VPC!Flow Logs}\index{Flow Logs (VPC)}} {\RVPCFlowLogs{\iconsize}} {Res\_Amazon-VPC\_Flow-Logs\_48\_Light.pdf} {{\textbackslash}RVPCFlowLogs} {VPC: Flow Logs}

\gxs{\href{https://www.google.com/search?q=AWS+VPC+Internet+Gateway}{VPC: Internet Gateway} \index{VPC!Internet Gateway}\index{Internet Gateway (VPC)}} {\RVPCInternetGateway{\iconsize}} {Res\_Amazon-VPC\_Internet-Gateway\_48\_Light.pdf} {{\textbackslash}RVPCInternetGateway} {VPC: Internet Gateway}

\gxs{\href{https://www.google.com/search?q=AWS+VPC+NAT+Gateway}{VPC: NAT Gateway} \index{VPC!NAT Gateway}\index{NAT Gateway (VPC)}} {\RVPCNATGateway{\iconsize}} {Res\_Amazon-VPC\_NAT-Gateway\_48\_Light.pdf} {{\textbackslash}RVPCNATGateway} {VPC: NAT Gateway}

\gxs{\href{https://www.google.com/search?q=AWS+VPC+Network+Access+Control+List}{VPC: Network Access Control List} \index{VPC!Network Access Control List}\index{Network Access Control List (VPC)}} {\RVPCNetworkAccessControlList{\iconsize}} {Res\_Amazon-VPC\_Network-Access-Control-List\_48\_Light.pdf} {{\textbackslash}RVPCNetworkAccessControlList} {VPC: Network Access Control List}

\gxs{\href{https://www.google.com/search?q=AWS+VPC+Peering+Connection}{VPC: Peering Connection} \index{VPC!Peering Connection}\index{Peering Connection (VPC)}} {\RVPCPeeringConnection{\iconsize}} {Res\_Amazon-VPC\_Peering-Connection\_48\_Light.pdf} {{\textbackslash}RVPCPeeringConnection} {VPC: Peering Connection}

\gxs{\href{https://www.google.com/search?q=AWS+VPC+Router}{VPC: Router} \index{VPC!Router}\index{Router (VPC)}} {\RVPCRouter{\iconsize}} {Res\_Amazon-VPC\_Router\_48\_Light.pdf} {{\textbackslash}RVPCRouter} {VPC: Router}

\gxs{\href{https://www.google.com/search?q=AWS+VPC+Traffic+Mirroring}{VPC: Traffic Mirroring} \index{VPC!Traffic Mirroring}\index{Traffic Mirroring (VPC)}} {\RVPCTrafficMirroring{\iconsize}} {Res\_Amazon-VPC\_Traffic-Mirroring\_48\_Light.pdf} {{\textbackslash}RVPCTrafficMirroring} {VPC: Traffic Mirroring}

\gxs{\href{https://www.google.com/search?q=AWS+VPC+VPN+Connection}{VPC: VPN Connection} \index{VPC!VPN Connection}\index{VPN Connection (VPC)}} {\RVPCVPNConnection{\iconsize}} {Res\_Amazon-VPC\_VPN-Connection\_48\_Light.pdf} {{\textbackslash}RVPCVPNConnection} {VPC: VPN Connection}

\gxs{\href{https://www.google.com/search?q=AWS+VPC+VPN+Gateway}{VPC: VPN Gateway} \index{VPC!VPN Gateway}\index{VPN Gateway (VPC)}} {\RVPCVPNGateway{\iconsize}} {Res\_Amazon-VPC\_VPN-Gateway\_48\_Light.pdf} {{\textbackslash}RVPCVPNGateway} {VPC: VPN Gateway}

\gxs{\href{https://www.google.com/search?q=AWS+WAF+Filtering+Rule}{WAF: Filtering Rule} \index{WAF!Filtering Rule}\index{Filtering Rule (WAF)}} {\RWAFFilteringRule{\iconsize}} {Res\_AWS-WAF\_Filtering-Rule\_48\_Light.pdf} {{\textbackslash}RWAFFilteringRule} {WAF: Filtering Rule}

\resEnd
\archStart
\gxs{\href{https://www.google.com/search?q=AWS+API+Gateway}{API Gateway} \index{API Gateway}} {\AAPIGateway{\iconsize}} {Arch\_Amazon-API-Gateway\_64.pdf} {{\textbackslash}AAPIGateway} {API Gateway}

\gxs{\href{https://www.google.com/search?q=AWS+Alexa+For+Business}{Alexa For Business} \index{Alexa For Business}} {\AAlexaForBusiness{\iconsize}} {Arch\_Alexa-For-Business\_64.pdf} {{\textbackslash}AAlexaForBusiness} {Alexa For Business}

\gxs{\href{https://www.google.com/search?q=AWS+Amplify+Console}{Amplify Console} \index{Amplify Console}} {\AAmplifyConsole{\iconsize}} {Arch\_AWS-Amplify-Console\_64.pdf} {{\textbackslash}AAmplifyConsole} {Amplify Console}

\gxs{\href{https://www.google.com/search?q=AWS+Apache+MXNet}{Apache MXNet} \index{Apache MXNet}} {\AApacheMXNet{\iconsize}} {Arch\_AWS-Apache-MXNet\_64.pdf} {{\textbackslash}AApacheMXNet} {Apache MXNet}

\gxs{\href{https://www.google.com/search?q=AWS+App+Config}{App Config} \index{App Config}} {\AAppConfig{\iconsize}} {Arch\_AWS-App-Config\_64.pdf} {{\textbackslash}AAppConfig} {App Config}

\gxs{\href{https://www.google.com/search?q=AWS+App+Mesh}{App Mesh} \index{App Mesh}} {\AAppMesh{\iconsize}} {Arch\_AWS-App-Mesh\_64.pdf} {{\textbackslash}AAppMesh} {App Mesh}

\gxs{\href{https://www.google.com/search?q=AWS+App+Stream}{App Stream} \index{App Stream}} {\AAppStream{\iconsize}} {Arch\_Amazon-App-Stream\_64.pdf} {{\textbackslash}AAppStream} {App Stream}

\gxs{\href{https://www.google.com/search?q=AWS+App+Wizard}{App Wizard} \index{App Wizard}} {\AAppWizard{\iconsize}} {Arch\_AWS-App-Wizard\_64.pdf} {{\textbackslash}AAppWizard} {App Wizard}

\gxs{\href{https://www.google.com/search?q=AWS+AppSync}{AppSync} \index{AppSync}} {\AAppSync{\iconsize}} {Arch\_AWS-AppSync\_64.pdf} {{\textbackslash}AAppSync} {AppSync}

\gxs{\href{https://www.google.com/search?q=AWS+Application+Auto+Scaling}{Application Auto Scaling} \index{Application Auto Scaling}} {\AAppAutoScaling{\iconsize}} {Arch\_Amazon-Application-Auto-Scaling\_64.pdf} {{\textbackslash}AAppAutoScaling} {Application Auto Scaling}

\gxs{\href{https://www.google.com/search?q=AWS+Application+Discovery+Service}{Application Discovery Service} \index{Application Discovery Service}} {\AADS{\iconsize}} {Arch\_AWS-Application-Discovery-Service\_64.pdf} {{\textbackslash}AADS} {Application Discovery Service}

\gxs{\href{https://www.google.com/search?q=AWS+Artifact}{Artifact} \index{Artifact}} {\AArtifact{\iconsize}} {Arch\_AWS-Artifact\_64.pdf} {{\textbackslash}AArtifact} {Artifact}

\gxs{\href{https://www.google.com/search?q=AWS+Athena}{Athena} \index{Athena}} {\AAthena{\iconsize}} {Arch\_Amazon-Athena\_64.pdf} {{\textbackslash}AAthena} {Athena}

\gxs{\href{https://www.google.com/search?q=AWS+Augmented+AI+A2I}{Augmented AI A2I} \index{Augmented AI A2I}} {\AAugmentedAIATwoI{\iconsize}} {Arch\_Amazon-Augmented-AI-A2I\_64.pdf} {{\textbackslash}AAugmentedAIATwoI} {Augmented AI A2I}

\gxs{\href{https://www.google.com/search?q=AWS+Aurora}{Aurora} \index{Aurora}} {\AAur{\iconsize}} {Arch\_Amazon-Aurora\_64.pdf} {{\textbackslash}AAur} {Aurora}

\gxs{\href{https://www.google.com/search?q=AWS+Auto+Scaling}{Auto Scaling} \index{Auto Scaling}} {\AAutoScaling{\iconsize}} {Arch\_AWS-Auto-Scaling\_64.pdf} {{\textbackslash}AAutoScaling} {Auto Scaling}

\gxs{\href{https://www.google.com/search?q=AWS+Backint+Agent}{Backint Agent} \index{Backint Agent}} {\ABackintAgent{\iconsize}} {Arch\_AWS-Backint-Agent\_64.pdf} {{\textbackslash}ABackintAgent} {Backint Agent}

\gxs{\href{https://www.google.com/search?q=AWS+Backup}{Backup} \index{Backup}} {\ABackup{\iconsize}} {Arch\_AWS-Backup\_64.pdf} {{\textbackslash}ABackup} {Backup}

\gxs{\href{https://www.google.com/search?q=AWS+Batch}{Batch} \index{Batch}} {\ABatch{\iconsize}} {Arch\_AWS-Batch\_64.pdf} {{\textbackslash}ABatch} {Batch}

\gxs{\href{https://www.google.com/search?q=AWS+Bottlerocket}{Bottlerocket} \index{Bottlerocket}} {\ABottlerocket{\iconsize}} {Arch\_AWS-Bottlerocket\_64.pdf} {{\textbackslash}ABottlerocket} {Bottlerocket}

\gxs{\href{https://www.google.com/search?q=AWS+Braket}{Braket} \index{Braket}} {\ABraket{\iconsize}} {Arch\_AWS-Braket\_64.pdf} {{\textbackslash}ABraket} {Braket}

\gxs{\href{https://www.google.com/search?q=AWS+Budgets}{Budgets} \index{Budgets}} {\ABudgets{\iconsize}} {Arch\_AWS-Budgets\_64.pdf} {{\textbackslash}ABudgets} {Budgets}

\gxs{\href{https://www.google.com/search?q=AWS+Certificate+Manager}{Certificate Manager} \index{Certificate Manager}} {\ACertMan{\iconsize}} {Arch\_AWS-Certificate-Manager\_64.pdf} {{\textbackslash}ACertMan} {Certificate Manager}

\gxs{\href{https://www.google.com/search?q=AWS+Chatbot}{Chatbot} \index{Chatbot}} {\AChatbot{\iconsize}} {Arch\_AWS-Chatbot\_64.pdf} {{\textbackslash}AChatbot} {Chatbot}

\gxs{\href{https://www.google.com/search?q=AWS+Chime}{Chime} \index{Chime}} {\AChime{\iconsize}} {Arch\_Amazon-Chime\_64.pdf} {{\textbackslash}AChime} {Chime}

\gxs{\href{https://www.google.com/search?q=AWS+Cloud+Development+Kit}{Cloud Development Kit} \index{Cloud Development Kit}} {\ACloudDevKit{\iconsize}} {Arch\_AWS-Cloud-Development-Kit\_64.pdf} {{\textbackslash}ACloudDevKit} {Cloud Development Kit}

\gxs{\href{https://www.google.com/search?q=AWS+Cloud+Directory}{Cloud Directory} \index{Cloud Directory}} {\ACloudDir{\iconsize}} {Arch\_Amazon-Cloud-Directory\_64.pdf} {{\textbackslash}ACloudDir} {Cloud Directory}

\gxs{\href{https://www.google.com/search?q=AWS+Cloud+Trail}{Cloud Trail} \index{Cloud Trail}} {\ACloudTrail{\iconsize}} {Arch\_AWS-Cloud-Trail\_64.pdf} {{\textbackslash}ACloudTrail} {Cloud Trail}

\gxs{\href{https://www.google.com/search?q=AWS+Cloud9}{Cloud9} \index{Cloud9}} {\ACloudNine{\iconsize}} {Arch\_AWS-Cloud9\_64.pdf} {{\textbackslash}ACloudNine} {Cloud9}

\gxs{\href{https://www.google.com/search?q=AWS+CloudEndure+Disaster+Recovery}{CloudEndure Disaster Recovery} \index{CloudEndure Disaster Recovery}} {\ACloudEndureDisasterRecov{\iconsize}} {Arch\_AWS-CloudEndure-Disaster-Recovery\_64.pdf} {{\textbackslash}ACloudEndureDisasterRecov} {CloudEndure Disaster Recovery}

\gxs{\href{https://www.google.com/search?q=AWS+CloudEndure+Migration}{CloudEndure Migration} \index{CloudEndure Migration}} {\ACloudEndureMigrat{\iconsize}} {Arch\_AWS-CloudEndure-Migration\_64.pdf} {{\textbackslash}ACloudEndureMigrat} {CloudEndure Migration}

\gxs{\href{https://www.google.com/search?q=AWS+CloudFormation}{CloudFormation} \index{CloudFormation}} {\ACloudForm{\iconsize}} {Arch\_AWS-CloudFormation\_64.pdf} {{\textbackslash}ACloudForm} {CloudFormation}

\gxs{\href{https://www.google.com/search?q=AWS+CloudFront}{CloudFront} \index{CloudFront}} {\ACloudFront{\iconsize}} {Arch\_Amazon-CloudFront\_64.pdf} {{\textbackslash}ACloudFront} {CloudFront}

\gxs{\href{https://www.google.com/search?q=AWS+CloudHSM}{CloudHSM} \index{CloudHSM}} {\ACloudHSM{\iconsize}} {Arch\_AWS-CloudHSM\_64.pdf} {{\textbackslash}ACloudHSM} {CloudHSM}

\gxs{\href{https://www.google.com/search?q=AWS+CloudMap}{CloudMap} \index{CloudMap}} {\ACloudMap{\iconsize}} {Arch\_AWS-CloudMap\_64.pdf} {{\textbackslash}ACloudMap} {CloudMap}

\gxs{\href{https://www.google.com/search?q=AWS+CloudSearch}{CloudSearch} \index{CloudSearch}} {\ACloudSearch{\iconsize}} {Arch\_Amazon-CloudSearch\_64.pdf} {{\textbackslash}ACloudSearch} {CloudSearch}

\gxs{\href{https://www.google.com/search?q=AWS+CloudWatch}{CloudWatch} \index{CloudWatch}} {\ACloudWatch{\iconsize}} {Arch\_Amazon-CloudWatch\_64.pdf} {{\textbackslash}ACloudWatch} {CloudWatch}

\gxs{\href{https://www.google.com/search?q=AWS+CodeArtifact}{CodeArtifact} \index{CodeArtifact}} {\ACodeArtifact{\iconsize}} {Arch\_AWS-CodeArtifact\_64.pdf} {{\textbackslash}ACodeArtifact} {CodeArtifact}

\gxs{\href{https://www.google.com/search?q=AWS+CodeBuild}{CodeBuild} \index{CodeBuild}} {\ACodeBuild{\iconsize}} {Arch\_Amazon-CodeBuild\_64.pdf} {{\textbackslash}ACodeBuild} {CodeBuild}

\gxs{\href{https://www.google.com/search?q=AWS+CodeCommit}{CodeCommit} \index{CodeCommit}} {\ACodeCommit{\iconsize}} {Arch\_AWS-CodeCommit\_64.pdf} {{\textbackslash}ACodeCommit} {CodeCommit}

\gxs{\href{https://www.google.com/search?q=AWS+CodeDeploy}{CodeDeploy} \index{CodeDeploy}} {\ACodeDeploy{\iconsize}} {Arch\_AWS-CodeDeploy\_64.pdf} {{\textbackslash}ACodeDeploy} {CodeDeploy}

\gxs{\href{https://www.google.com/search?q=AWS+CodeGuru}{CodeGuru} \index{CodeGuru}} {\ACodeGuru{\iconsize}} {Arch\_Amazon-CodeGuru\_64.pdf} {{\textbackslash}ACodeGuru} {CodeGuru}

\gxs{\href{https://www.google.com/search?q=AWS+CodePipeline}{CodePipeline} \index{CodePipeline}} {\ACodePipeline{\iconsize}} {Arch\_AWS-CodePipeline\_64.pdf} {{\textbackslash}ACodePipeline} {CodePipeline}

\gxs{\href{https://www.google.com/search?q=AWS+Codestar}{Codestar} \index{Codestar}} {\ACodestar{\iconsize}} {Arch\_Amazon-Codestar\_64.pdf} {{\textbackslash}ACodestar} {Codestar}

\gxs{\href{https://www.google.com/search?q=AWS+Cognito}{Cognito} \index{Cognito}} {\ACognito{\iconsize}} {Arch\_Amazon-Cognito\_64.pdf} {{\textbackslash}ACognito} {Cognito}

\gxs{\href{https://www.google.com/search?q=AWS+Command+Line+Interface}{Command Line Interface} \index{Command Line Interface}} {\ACLI{\iconsize}} {Arch\_AWS-Command-Line-Interface\_64.pdf} {{\textbackslash}ACLI} {Command Line Interface}

\gxs{\href{https://www.google.com/search?q=AWS+Comprehend}{Comprehend} \index{Comprehend}} {\AComprehend{\iconsize}} {Arch\_Amazon-Comprehend\_64.pdf} {{\textbackslash}AComprehend} {Comprehend}

\gxs{\href{https://www.google.com/search?q=AWS+Compute+Optimizer}{Compute Optimizer} \index{Compute Optimizer}} {\ACmputOptimizer{\iconsize}} {Arch\_AWS-Compute-Optimizer\_64.pdf} {{\textbackslash}ACmputOptimizer} {Compute Optimizer}

\gxs{\href{https://www.google.com/search?q=AWS+Config}{Config} \index{Config}} {\AConfig{\iconsize}} {Arch\_AWS-Config\_64.pdf} {{\textbackslash}AConfig} {Config}

\gxs{\href{https://www.google.com/search?q=AWS+Connect}{Connect} \index{Connect}} {\AConnect{\iconsize}} {Arch\_Amazon-Connect\_64.pdf} {{\textbackslash}AConnect} {Connect}

\gxs{\href{https://www.google.com/search?q=AWS+Console+Mobile+Application}{Console Mobile Application} \index{Console Mobile Application}} {\AConsoleMobileApp{\iconsize}} {Arch\_AWS-Console-Mobile-Application\_64.pdf} {{\textbackslash}AConsoleMobileApp} {Console Mobile Application}

\gxs{\href{https://www.google.com/search?q=AWS+Control+Tower}{Control Tower} \index{Control Tower}} {\AControlTower{\iconsize}} {Arch\_AWS-Control-Tower\_64.pdf} {{\textbackslash}AControlTower} {Control Tower}

\gxs{\href{https://www.google.com/search?q=AWS+Cost+Explorer}{Cost Explorer} \index{Cost Explorer}} {\ACostExplorer{\iconsize}} {Arch\_AWS-Cost-Explorer\_64.pdf} {{\textbackslash}ACostExplorer} {Cost Explorer}

\gxs{\href{https://www.google.com/search?q=AWS+Cost+and+Usage+Report}{Cost and Usage Report} \index{Cost and Usage Report}} {\ACostUseRep{\iconsize}} {Arch\_AWS-Cost-and-Usage-Report\_64.pdf} {{\textbackslash}ACostUseRep} {Cost and Usage Report}

\gxs{\href{https://www.google.com/search?q=AWS+Data+Exchange}{Data Exchange} \index{Data Exchange}} {\ADataExchange{\iconsize}} {Arch\_AWS-Data-Exchange\_64.pdf} {{\textbackslash}ADataExchange} {Data Exchange}

\gxs{\href{https://www.google.com/search?q=AWS+Data+Pipeline}{Data Pipeline} \index{Data Pipeline}} {\ADataPipeline{\iconsize}} {Arch\_AWS-Data-Pipeline\_64.pdf} {{\textbackslash}ADataPipeline} {Data Pipeline}

\gxs{\href{https://www.google.com/search?q=AWS+Data+Sync}{Data Sync} \index{Data Sync}} {\ADataSync{\iconsize}} {Arch\_AWS-Data-Sync\_64.pdf} {{\textbackslash}ADataSync} {Data Sync}

\gxs{\href{https://www.google.com/search?q=AWS+Database+Migration}{Database Migration} \index{Database Migration}} {\ADBMigrat{\iconsize}} {Arch\_AWS-Database-Migration\_64.pdf} {{\textbackslash}ADBMigrat} {Database Migration}

\gxs{\href{https://www.google.com/search?q=AWS+Deep+Composer}{Deep Composer} \index{Deep Composer}} {\ADeepComposer{\iconsize}} {Arch\_AWS-Deep-Composer\_64.pdf} {{\textbackslash}ADeepComposer} {Deep Composer}

\gxs{\href{https://www.google.com/search?q=AWS+Deep+Learning+AMIs}{Deep Learning AMIs} \index{Deep Learning AMIs}} {\ADeepLearningAMIs{\iconsize}} {Arch\_AWS-Deep-Learning-AMIs\_64.pdf} {{\textbackslash}ADeepLearningAMIs} {Deep Learning AMIs}

\gxs{\href{https://www.google.com/search?q=AWS+Deep+Learning+Containers}{Deep Learning Containers} \index{Deep Learning Containers}} {\ADeepLearningContainers{\iconsize}} {Arch\_AWS-Deep-Learning-Containers\_64.pdf} {{\textbackslash}ADeepLearningContainers} {Deep Learning Containers}

\gxs{\href{https://www.google.com/search?q=AWS+Deep+Lense}{Deep Lense} \index{Deep Lense}} {\ADeepLense{\iconsize}} {Arch\_AWS-Deep-Lense\_64.pdf} {{\textbackslash}ADeepLense} {Deep Lense}

\gxs{\href{https://www.google.com/search?q=AWS+Deepracer}{Deepracer} \index{Deepracer}} {\ADeepracer{\iconsize}} {Arch\_AWS-Deepracer\_64.pdf} {{\textbackslash}ADeepracer} {Deepracer}

\gxs{\href{https://www.google.com/search?q=AWS+Detective}{Detective} \index{Detective}} {\ADetective{\iconsize}} {Arch\_Amazon-Detective\_64.pdf} {{\textbackslash}ADetective} {Detective}

\gxs{\href{https://www.google.com/search?q=AWS+Device+Farm}{Device Farm} \index{Device Farm}} {\ADevFarm{\iconsize}} {Arch\_AWS-Device-Farm\_64.pdf} {{\textbackslash}ADevFarm} {Device Farm}

\gxs{\href{https://www.google.com/search?q=AWS+Direct+Connect}{Direct Connect} \index{Direct Connect}} {\ADirectConnect{\iconsize}} {Arch\_AWS-Direct-Connect\_64.pdf} {{\textbackslash}ADirectConnect} {Direct Connect}

\gxs{\href{https://www.google.com/search?q=AWS+Directory+Service}{Directory Service} \index{Directory Service}} {\ADirSvc{\iconsize}} {Arch\_AWS-Directory-Service\_64.pdf} {{\textbackslash}ADirSvc} {Directory Service}

\gxs{\href{https://www.google.com/search?q=AWS+DocumentDB}{DocumentDB} \index{DocumentDB}} {\ADocumentDB{\iconsize}} {Arch\_Amazon-DocumentDB\_64.pdf} {{\textbackslash}ADocumentDB} {DocumentDB}

\gxs{\href{https://www.google.com/search?q=AWS+DynamoDB}{DynamoDB} \index{DynamoDB}} {\ADDB{\iconsize}} {Arch\_Amazon-DynamoDB\_64.pdf} {{\textbackslash}ADDB} {DynamoDB}

\gxs{\href{https://www.google.com/search?q=AWS+EC2}{EC2} \index{EC2}} {\AECTwo{\iconsize}} {Arch\_Amazon-EC2\_64.pdf} {{\textbackslash}AECTwo} {EC2}

\gxs{\href{https://www.google.com/search?q=AWS+EC2+Auto+Scaling}{EC2 Auto Scaling} \index{EC2 Auto Scaling}} {\AECTwoAutoScaling{\iconsize}} {Arch\_Amazon-EC2-Auto-Scaling\_64.pdf} {{\textbackslash}AECTwoAutoScaling} {EC2 Auto Scaling}

\gxs{\href{https://www.google.com/search?q=AWS+EC2+Image+Builder}{EC2 Image Builder} \index{EC2 Image Builder}} {\AECTwoImgBuilder{\iconsize}} {Arch\_AWS-EC2-Image-Builder\_64.pdf} {{\textbackslash}AECTwoImgBuilder} {EC2 Image Builder}

\gxs{\href{https://www.google.com/search?q=AWS+EC2+M5n}{EC2 M5n} \index{EC2 M5n}} {\AECTwoMFiven{\iconsize}} {Arch\_Amazon-EC2-M5n\_64.pdf} {{\textbackslash}AECTwoMFiven} {EC2 M5n}

\gxs{\href{https://www.google.com/search?q=AWS+EC2+R5n}{EC2 R5n} \index{EC2 R5n}} {\AECTwoRFiven{\iconsize}} {Arch\_Amazon-EC2-R5n\_64.pdf} {{\textbackslash}AECTwoRFiven} {EC2 R5n}

\gxs{\href{https://www.google.com/search?q=AWS+EMR}{EMR} \index{EMR}} {\AEMR{\iconsize}} {Arch\_Amazon-EMR\_64.pdf} {{\textbackslash}AEMR} {EMR}

\gxs{\href{https://www.google.com/search?q=AWS+ElastiCache}{ElastiCache} \index{ElastiCache}} {\AElastiCache{\iconsize}} {Arch\_Amazon-ElastiCache\_64.pdf} {{\textbackslash}AElastiCache} {ElastiCache}

\gxs{\href{https://www.google.com/search?q=AWS+Elastic+Beanstalk}{Elastic Beanstalk} \index{Elastic Beanstalk}} {\AEB{\iconsize}} {Arch\_AWS-Elastic-Beanstalk\_64.pdf} {{\textbackslash}AEB} {Elastic Beanstalk}

\gxs{\href{https://www.google.com/search?q=AWS+Elastic+Block+Store}{Elastic Block Store} \index{Elastic Block Store}} {\AEBS{\iconsize}} {Arch\_Amazon-Elastic-Block-Store\_64.pdf} {{\textbackslash}AEBS} {Elastic Block Store}

\gxs{\href{https://www.google.com/search?q=AWS+Elastic+Container+Kubernetes}{Elastic Container Kubernetes} \index{Elastic Container Kubernetes}} {\AECK{\iconsize}} {Arch\_Amazon-Elastic-Container-Kubernetes\_64.pdf} {{\textbackslash}AECK} {Elastic Container Kubernetes}

\gxs{\href{https://www.google.com/search?q=AWS+Elastic+Container+Registry}{Elastic Container Registry} \index{Elastic Container Registry}} {\AECR{\iconsize}} {Arch\_Amazon-Elastic-Container-Registry\_64.pdf} {{\textbackslash}AECR} {Elastic Container Registry}

\gxs{\href{https://www.google.com/search?q=AWS+Elastic+Container+Service}{Elastic Container Service} \index{Elastic Container Service}} {\AECS{\iconsize}} {Arch\_Amazon-Elastic-Container-Service\_64.pdf} {{\textbackslash}AECS} {Elastic Container Service}

\gxs{\href{https://www.google.com/search?q=AWS+Elastic+File+System}{Elastic File System} \index{Elastic File System}} {\AEFS{\iconsize}} {Arch\_Amazon-Elastic-File-System\_64.pdf} {{\textbackslash}AEFS} {Elastic File System}

\gxs{\href{https://www.google.com/search?q=AWS+Elastic+Inference}{Elastic Inference} \index{Elastic Inference}} {\AElasticInference{\iconsize}} {Arch\_Amazon-Elastic-Inference\_64.pdf} {{\textbackslash}AElasticInference} {Elastic Inference}

\gxs{\href{https://www.google.com/search?q=AWS+Elastic+Load+Balancing}{Elastic Load Balancing} \index{Elastic Load Balancing}} {\AELB{\iconsize}} {Arch\_Elastic-Load-Balancing\_64.pdf} {{\textbackslash}AELB} {Elastic Load Balancing}

\gxs{\href{https://www.google.com/search?q=AWS+Elastic+Transcoder}{Elastic Transcoder} \index{Elastic Transcoder}} {\AElasticTranscoder{\iconsize}} {Arch\_Amazon-Elastic-Transcoder\_64.pdf} {{\textbackslash}AElasticTranscoder} {Elastic Transcoder}

\gxs{\href{https://www.google.com/search?q=AWS+Elasticsearch+Service}{Elasticsearch Service} \index{Elasticsearch Service}} {\AElasticsearchSvc{\iconsize}} {Arch\_Amazon-Elasticsearch-Service\_64.pdf} {{\textbackslash}AElasticsearchSvc} {Elasticsearch Service}

\gxs{\href{https://www.google.com/search?q=AWS+Elemental+Appliances+And+Software}{Elemental Appliances And Software} \index{Elemental Appliances And Software}} {\AElemAppliancesAndSoftware{\iconsize}} {Arch\_AWS-Elemental-Appliances-And-Software\_64.pdf} {{\textbackslash}AElemAppliancesAndSoftware} {Elemental Appliances And Software}

\gxs{\href{https://www.google.com/search?q=AWS+Elemental+Link}{Elemental Link} \index{Elemental Link}} {\AElemLink{\iconsize}} {Arch\_AWS-Elemental-Link\_64.pdf} {{\textbackslash}AElemLink} {Elemental Link}

\gxs{\href{https://www.google.com/search?q=AWS+Elemental+MediaConnect}{Elemental MediaConnect} \index{Elemental MediaConnect}} {\AElemMediaConnect{\iconsize}} {Arch\_AWS-Elemental-MediaConnect\_64.pdf} {{\textbackslash}AElemMediaConnect} {Elemental MediaConnect}

\gxs{\href{https://www.google.com/search?q=AWS+Elemental+MediaConvert}{Elemental MediaConvert} \index{Elemental MediaConvert}} {\AElemMediaConvert{\iconsize}} {Arch\_AWS-Elemental-MediaConvert\_64.pdf} {{\textbackslash}AElemMediaConvert} {Elemental MediaConvert}

\gxs{\href{https://www.google.com/search?q=AWS+Elemental+MediaLive}{Elemental MediaLive} \index{Elemental MediaLive}} {\AElemMediaLive{\iconsize}} {Arch\_AWS-Elemental-MediaLive\_64.pdf} {{\textbackslash}AElemMediaLive} {Elemental MediaLive}

\gxs{\href{https://www.google.com/search?q=AWS+Elemental+MediaPackage}{Elemental MediaPackage} \index{Elemental MediaPackage}} {\AElemMediaPackage{\iconsize}} {Arch\_AWS-Elemental-MediaPackage\_64.pdf} {{\textbackslash}AElemMediaPackage} {Elemental MediaPackage}

\gxs{\href{https://www.google.com/search?q=AWS+Elemental+MediaStore}{Elemental MediaStore} \index{Elemental MediaStore}} {\AElemMediaStore{\iconsize}} {Arch\_AWS-Elemental-MediaStore\_64.pdf} {{\textbackslash}AElemMediaStore} {Elemental MediaStore}

\gxs{\href{https://www.google.com/search?q=AWS+Elemental+MediaTailor}{Elemental MediaTailor} \index{Elemental MediaTailor}} {\AElemMediaTailor{\iconsize}} {Arch\_AWS-Elemental-MediaTailor\_64.pdf} {{\textbackslash}AElemMediaTailor} {Elemental MediaTailor}

\gxs{\href{https://www.google.com/search?q=AWS+EventBridge}{EventBridge} \index{EventBridge}} {\AEvBr{\iconsize}} {Arch\_Amazon-EventBridge\_64.pdf} {{\textbackslash}AEvBr} {EventBridge}

\gxs{\href{https://www.google.com/search?q=AWS+Express+Workflow}{Express Workflow} \index{Express Workflow}} {\AExpressWorkflow{\iconsize}} {Arch\_AWS-Express-Workflow\_64.pdf} {{\textbackslash}AExpressWorkflow} {Express Workflow}

\gxs{\href{https://www.google.com/search?q=AWS+FSx}{FSx} \index{FSx}} {\AFSx{\iconsize}} {Arch\_Amazon-FSx\_64.pdf} {{\textbackslash}AFSx} {FSx}

\gxs{\href{https://www.google.com/search?q=AWS+FSx+For+WFS}{FSx For WFS} \index{FSx For WFS}} {\AFSxForWFS{\iconsize}} {Arch\_Amazon-FSx-For-WFS\_64.pdf} {{\textbackslash}AFSxForWFS} {FSx For WFS}

\gxs{\href{https://www.google.com/search?q=AWS+FSx+for+Lustre}{FSx for Lustre} \index{FSx for Lustre}} {\AFSxforLustre{\iconsize}} {Arch\_Amazon-FSx-for-Lustre\_64.pdf} {{\textbackslash}AFSxforLustre} {FSx for Lustre}

\gxs{\href{https://www.google.com/search?q=AWS+Fargate}{Fargate} \index{Fargate}} {\AFargate{\iconsize}} {Arch\_AWS-Fargate\_64.pdf} {{\textbackslash}AFargate} {Fargate}

\gxs{\href{https://www.google.com/search?q=AWS+Firewall+Manager}{Firewall Manager} \index{Firewall Manager}} {\AFirewallMngr{\iconsize}} {Arch\_AWS-Firewall-Manager\_64.pdf} {{\textbackslash}AFirewallMngr} {Firewall Manager}

\gxs{\href{https://www.google.com/search?q=AWS+Forecast}{Forecast} \index{Forecast}} {\AForecast{\iconsize}} {Arch\_Amazon-Forecast\_64.pdf} {{\textbackslash}AForecast} {Forecast}

\gxs{\href{https://www.google.com/search?q=AWS+Fraud+Detector}{Fraud Detector} \index{Fraud Detector}} {\AFraudDetector{\iconsize}} {Arch\_Amazon-Fraud-Detector\_64.pdf} {{\textbackslash}AFraudDetector} {Fraud Detector}

\gxs{\href{https://www.google.com/search?q=AWS+FreeRTOS}{FreeRTOS} \index{FreeRTOS}} {\AFreeRTOS{\iconsize}} {Arch\_Amazon-FreeRTOS\_64.pdf} {{\textbackslash}AFreeRTOS} {FreeRTOS}

\gxs{\href{https://www.google.com/search?q=AWS+GameLift}{GameLift} \index{GameLift}} {\AGameLift{\iconsize}} {Arch\_Amazon-GameLift\_64.pdf} {{\textbackslash}AGameLift} {GameLift}

\gxs{\href{https://www.google.com/search?q=AWS+Glacier}{Glacier} \index{Glacier}} {\AGlacier{\iconsize}} {Arch\_Amazon-Glacier\_64.pdf} {{\textbackslash}AGlacier} {Glacier}

\gxs{\href{https://www.google.com/search?q=AWS+Global+Accelerator}{Global Accelerator} \index{Global Accelerator}} {\AGlblAccelerator{\iconsize}} {Arch\_AWS-Global-Accelerator\_64.pdf} {{\textbackslash}AGlblAccelerator} {Global Accelerator}

\gxs{\href{https://www.google.com/search?q=AWS+Glue}{Glue} \index{Glue}} {\AGlue{\iconsize}} {Arch\_AWS-Glue\_64.pdf} {{\textbackslash}AGlue} {Glue}

\gxs{\href{https://www.google.com/search?q=AWS+Ground+Station}{Ground Station} \index{Ground Station}} {\AGroundStation{\iconsize}} {Arch\_AWS-Ground-Station\_64.pdf} {{\textbackslash}AGroundStation} {Ground Station}

\gxs{\href{https://www.google.com/search?q=AWS+GuardDuty}{GuardDuty} \index{GuardDuty}} {\AGuardDuty{\iconsize}} {Arch\_Amazon-GuardDuty\_64.pdf} {{\textbackslash}AGuardDuty} {GuardDuty}

\gxs{\href{https://www.google.com/search?q=AWS+IQ}{IQ} \index{IQ}} {\AIQ{\iconsize}} {Arch\_AWS-IQ\_64.pdf} {{\textbackslash}AIQ} {IQ}

\gxs{\href{https://www.google.com/search?q=AWS+Identity+and+Access+Management}{Identity and Access Management} \index{Identity and Access Management}} {\AIAM{\iconsize}} {Arch\_AWS-Identity-and-Access-Management\_64.pdf} {{\textbackslash}AIAM} {Identity and Access Management}

\gxs{\href{https://www.google.com/search?q=AWS+Infrequent+Access+Storage+Class}{Infrequent Access Storage Class} \index{Infrequent Access Storage Class}} {\AInfrequentAccessStorageClass{\iconsize}} {Arch\_Infrequent-Access-Storage-Class\_64.pdf} {{\textbackslash}AInfrequentAccessStorageClass} {Infrequent Access Storage Class}

\gxs{\href{https://www.google.com/search?q=AWS+Inspector}{Inspector} \index{Inspector}} {\AInspector{\iconsize}} {Arch\_Amazon-Inspector\_64.pdf} {{\textbackslash}AInspector} {Inspector}

\gxs{\href{https://www.google.com/search?q=AWS+Interactive+Video}{Interactive Video} \index{Interactive Video}} {\AInteractiveVideo{\iconsize}} {Arch\_Amazon-Interactive-Video\_64.pdf} {{\textbackslash}AInteractiveVideo} {Interactive Video}

\gxs{\href{https://www.google.com/search?q=AWS+IoT+1+Click}{IoT 1 Click} \index{IoT 1 Click}} {\AIoTOneClick{\iconsize}} {Arch\_AWS-IoT-1-Click\_64.pdf} {{\textbackslash}AIoTOneClick} {IoT 1 Click}

\gxs{\href{https://www.google.com/search?q=AWS+IoT+Analytics}{IoT Analytics} \index{IoT Analytics}} {\AIoTAnalytics{\iconsize}} {Arch\_AWS-IoT-Analytics\_64.pdf} {{\textbackslash}AIoTAnalytics} {IoT Analytics}

\gxs{\href{https://www.google.com/search?q=AWS+IoT+Button}{IoT Button} \index{IoT Button}} {\AIoTButton{\iconsize}} {Arch\_AWS-IoT-Button\_64.pdf} {{\textbackslash}AIoTButton} {IoT Button}

\gxs{\href{https://www.google.com/search?q=AWS+IoT+Core}{IoT Core} \index{IoT Core}} {\AIoTCore{\iconsize}} {Arch\_AWS-IoT-Core\_64.pdf} {{\textbackslash}AIoTCore} {IoT Core}

\gxs{\href{https://www.google.com/search?q=AWS+IoT+Device+Defender}{IoT Device Defender} \index{IoT Device Defender}} {\AIoTDevDefender{\iconsize}} {Arch\_AWS-IoT-Device-Defender\_64.pdf} {{\textbackslash}AIoTDevDefender} {IoT Device Defender}

\gxs{\href{https://www.google.com/search?q=AWS+IoT+Device+Management}{IoT Device Management} \index{IoT Device Management}} {\AIoTDevManagement{\iconsize}} {Arch\_AWS-IoT-Device-Management\_64.pdf} {{\textbackslash}AIoTDevManagement} {IoT Device Management}

\gxs{\href{https://www.google.com/search?q=AWS+IoT+Events}{IoT Events} \index{IoT Events}} {\AIoTEvents{\iconsize}} {Arch\_AWS-IoT-Events\_64.pdf} {{\textbackslash}AIoTEvents} {IoT Events}

\gxs{\href{https://www.google.com/search?q=AWS+IoT+Greengrass+Core}{IoT Greengrass Core} \index{IoT Greengrass Core}} {\AIoTGreengrassCore{\iconsize}} {Arch\_AWS-IoT-Greengrass-Core\_64.pdf} {{\textbackslash}AIoTGreengrassCore} {IoT Greengrass Core}

\gxs{\href{https://www.google.com/search?q=AWS+IoT+SiteWise}{IoT SiteWise} \index{IoT SiteWise}} {\AIoTSiteWise{\iconsize}} {Arch\_AWS-IoT-SiteWise\_64.pdf} {{\textbackslash}AIoTSiteWise} {IoT SiteWise}

\gxs{\href{https://www.google.com/search?q=AWS+IoT+Things+Graph}{IoT Things Graph} \index{IoT Things Graph}} {\AIoTThngsGraph{\iconsize}} {Arch\_AWS-IoT-Things-Graph\_64.pdf} {{\textbackslash}AIoTThngsGraph} {IoT Things Graph}

\gxs{\href{https://www.google.com/search?q=AWS+Kendra}{Kendra} \index{Kendra}} {\AKendra{\iconsize}} {Arch\_Amazon-Kendra\_64.pdf} {{\textbackslash}AKendra} {Kendra}

\gxs{\href{https://www.google.com/search?q=AWS+Key+Management+Services}{Key Management Services} \index{Key Management Services}} {\AKMS{\iconsize}} {Arch\_AWS-Key-Management-Services\_64.pdf} {{\textbackslash}AKMS} {Key Management Services}

\gxs{\href{https://www.google.com/search?q=AWS+Keyspaces}{Keyspaces} \index{Keyspaces}} {\AKeyspaces{\iconsize}} {Arch\_Amazon-Keyspaces\_64.pdf} {{\textbackslash}AKeyspaces} {Keyspaces}

\gxs{\href{https://www.google.com/search?q=AWS+Kinesis}{Kinesis} \index{Kinesis}} {\AKin{\iconsize}} {Arch\_Amazon-Kinesis\_64.pdf} {{\textbackslash}AKin} {Kinesis}

\gxs{\href{https://www.google.com/search?q=AWS+Kinesis+Data+Analytics}{Kinesis Data Analytics} \index{Kinesis Data Analytics}} {\AKinDataAnalytics{\iconsize}} {Arch\_Amazon-Kinesis-Data-Analytics\_64.pdf} {{\textbackslash}AKinDataAnalytics} {Kinesis Data Analytics}

\gxs{\href{https://www.google.com/search?q=AWS+Kinesis+Data+Streams}{Kinesis Data Streams} \index{Kinesis Data Streams}} {\AKinDataStreams{\iconsize}} {Arch\_Amazon-Kinesis-Data-Streams\_64.pdf} {{\textbackslash}AKinDataStreams} {Kinesis Data Streams}

\gxs{\href{https://www.google.com/search?q=AWS+Kinesis+Firehose}{Kinesis Firehose} \index{Kinesis Firehose}} {\AKinFirehose{\iconsize}} {Arch\_Amazon-Kinesis-Firehose\_64.pdf} {{\textbackslash}AKinFirehose} {Kinesis Firehose}

\gxs{\href{https://www.google.com/search?q=AWS+Kinesis+Video+Streams}{Kinesis Video Streams} \index{Kinesis Video Streams}} {\AKinVideoStreams{\iconsize}} {Arch\_Amazon-Kinesis-Video-Streams\_64.pdf} {{\textbackslash}AKinVideoStreams} {Kinesis Video Streams}

\gxs{\href{https://www.google.com/search?q=AWS+Lake+Formation}{Lake Formation} \index{Lake Formation}} {\ALakeFormation{\iconsize}} {Arch\_AWS-Lake-Formation\_64.pdf} {{\textbackslash}ALakeFormation} {Lake Formation}

\gxs{\href{https://www.google.com/search?q=AWS+Lambda}{Lambda} \index{Lambda}} {\ALambda{\iconsize}} {Arch\_AWS-Lambda\_64.pdf} {{\textbackslash}ALambda} {Lambda}

\gxs{\href{https://www.google.com/search?q=AWS+Lex}{Lex} \index{Lex}} {\ALex{\iconsize}} {Arch\_Amazon-Lex\_64.pdf} {{\textbackslash}ALex} {Lex}

\gxs{\href{https://www.google.com/search?q=AWS+License+Manager}{License Manager} \index{License Manager}} {\ALicnsMngr{\iconsize}} {Arch\_AWS-License-Manager\_64.pdf} {{\textbackslash}ALicnsMngr} {License Manager}

\gxs{\href{https://www.google.com/search?q=AWS+Lightsail}{Lightsail} \index{Lightsail}} {\ALightsail{\iconsize}} {Arch\_Amazon-Lightsail\_64.pdf} {{\textbackslash}ALightsail} {Lightsail}

\gxs{\href{https://www.google.com/search?q=AWS+Local+Zones}{Local Zones} \index{Local Zones}} {\ALocalZones{\iconsize}} {Arch\_AWS-Local-Zones\_64.pdf} {{\textbackslash}ALocalZones} {Local Zones}

\gxs{\href{https://www.google.com/search?q=AWS+Lumberyard}{Lumberyard} \index{Lumberyard}} {\ALumberyard{\iconsize}} {Arch\_Amazon-Lumberyard\_64.pdf} {{\textbackslash}ALumberyard} {Lumberyard}

\gxs{\href{https://www.google.com/search?q=AWS+MQ}{MQ} \index{MQ}} {\AMQ{\iconsize}} {Arch\_Amazon-MQ\_64.pdf} {{\textbackslash}AMQ} {MQ}

\gxs{\href{https://www.google.com/search?q=AWS+Macie}{Macie} \index{Macie}} {\AMacie{\iconsize}} {Arch\_Amazon-Macie\_64.pdf} {{\textbackslash}AMacie} {Macie}

\gxs{\href{https://www.google.com/search?q=AWS+Managed+Blockchain}{Managed Blockchain} \index{Managed Blockchain}} {\AMngdBlockchain{\iconsize}} {Arch\_Amazon-Managed-Blockchain\_64.pdf} {{\textbackslash}AMngdBlockchain} {Managed Blockchain}

\gxs{\href{https://www.google.com/search?q=AWS+Managed+Services}{Managed Services} \index{Managed Services}} {\AMngdSvcs{\iconsize}} {Arch\_AWS-Managed-Services\_64.pdf} {{\textbackslash}AMngdSvcs} {Managed Services}

\gxs{\href{https://www.google.com/search?q=AWS+Managed+Streaming+for+Kafka}{Managed Streaming for Kafka} \index{Managed Streaming for Kafka}} {\AMngdStreamingforKafka{\iconsize}} {Arch\_Amazon-Managed-Streaming-for-Kafka\_64.pdf} {{\textbackslash}AMngdStreamingforKafka} {Managed Streaming for Kafka}

\gxs{\href{https://www.google.com/search?q=AWS+Management+Console}{Management Console} \index{Management Console}} {\AManagementConsole{\iconsize}} {Arch\_AWS-Management-Console\_64.pdf} {{\textbackslash}AManagementConsole} {Management Console}

\gxs{\href{https://www.google.com/search?q=AWS+Migration+Hub}{Migration Hub} \index{Migration Hub}} {\AMigratHub{\iconsize}} {Arch\_AWS-Migration-Hub\_64.pdf} {{\textbackslash}AMigratHub} {Migration Hub}

\gxs{\href{https://www.google.com/search?q=AWS+Neptune}{Neptune} \index{Neptune}} {\ANeptune{\iconsize}} {Arch\_Amazon-Neptune\_64.pdf} {{\textbackslash}ANeptune} {Neptune}

\gxs{\href{https://www.google.com/search?q=AWS+Neuron+ML+SDK}{Neuron ML SDK} \index{Neuron ML SDK}} {\ANeuronMLSDK{\iconsize}} {Arch\_Amazon-Neuron-ML-SDK\_64.pdf} {{\textbackslash}ANeuronMLSDK} {Neuron ML SDK}

\gxs{\href{https://www.google.com/search?q=AWS+Nitro+Enclaves}{Nitro Enclaves} \index{Nitro Enclaves}} {\ANitroEnclaves{\iconsize}} {Arch\_AWS-Nitro-Enclaves\_64.pdf} {{\textbackslash}ANitroEnclaves} {Nitro Enclaves}

\gxs{\href{https://www.google.com/search?q=AWS+OpsWorks}{OpsWorks} \index{OpsWorks}} {\AOpWk{\iconsize}} {Arch\_AWS-OpsWorks\_64.pdf} {{\textbackslash}AOpWk} {OpsWorks}

\gxs{\href{https://www.google.com/search?q=AWS+Organizations}{Organizations} \index{Organizations}} {\AOrganizations{\iconsize}} {Arch\_AWS-Organizations\_64.pdf} {{\textbackslash}AOrganizations} {Organizations}

\gxs{\href{https://www.google.com/search?q=AWS+Outposts}{Outposts} \index{Outposts}} {\AOutposts{\iconsize}} {Arch\_AWS-Outposts\_64.pdf} {{\textbackslash}AOutposts} {Outposts}

\gxs{\href{https://www.google.com/search?q=AWS+Parallel+Cluster}{Parallel Cluster} \index{Parallel Cluster}} {\AParallelCluster{\iconsize}} {Arch\_AWS-Parallel-Cluster\_64.pdf} {{\textbackslash}AParallelCluster} {Parallel Cluster}

\gxs{\href{https://www.google.com/search?q=AWS+Personal+Health+Dashboard}{Personal Health Dashboard} \index{Personal Health Dashboard}} {\APersHlthDbrd{\iconsize}} {Arch\_AWS-Personal-Health-Dashboard\_64.pdf} {{\textbackslash}APersHlthDbrd} {Personal Health Dashboard}

\gxs{\href{https://www.google.com/search?q=AWS+Personalize}{Personalize} \index{Personalize}} {\APersonalize{\iconsize}} {Arch\_Amazon-Personalize\_64.pdf} {{\textbackslash}APersonalize} {Personalize}

\gxs{\href{https://www.google.com/search?q=AWS+Pinpoint}{Pinpoint} \index{Pinpoint}} {\APinpoint{\iconsize}} {Arch\_Amazon-Pinpoint\_64.pdf} {{\textbackslash}APinpoint} {Pinpoint}

\gxs{\href{https://www.google.com/search?q=AWS+Pinpoint+Journey}{Pinpoint Journey} \index{Pinpoint Journey}} {\APinpointJourney{\iconsize}} {Arch\_Amazon-Pinpoint-Journey\_64.pdf} {{\textbackslash}APinpointJourney} {Pinpoint Journey}

\gxs{\href{https://www.google.com/search?q=AWS+Polly}{Polly} \index{Polly}} {\APolly{\iconsize}} {Arch\_Amazon-Polly\_64.pdf} {{\textbackslash}APolly} {Polly}

\gxs{\href{https://www.google.com/search?q=AWS+PrivateLink}{PrivateLink} \index{PrivateLink}} {\APrivateLink{\iconsize}} {Arch\_AWS-PrivateLink\_64.pdf} {{\textbackslash}APrivateLink} {PrivateLink}

\gxs{\href{https://www.google.com/search?q=AWS+Professional+Services}{Professional Services} \index{Professional Services}} {\AProfessionalSvcs{\iconsize}} {Arch\_AWS-Professional-Services\_64.pdf} {{\textbackslash}AProfessionalSvcs} {Professional Services}

\gxs{\href{https://www.google.com/search?q=AWS+Quantum+Ledger+Database}{Quantum Ledger Database} \index{Quantum Ledger Database}} {\AQuantumLedgerDB{\iconsize}} {Arch\_Amazon-Quantum-Ledger-Database\_64.pdf} {{\textbackslash}AQuantumLedgerDB} {Quantum Ledger Database}

\gxs{\href{https://www.google.com/search?q=AWS+QuickSight}{QuickSight} \index{QuickSight}} {\AQuickSight{\iconsize}} {Arch\_Amazon-QuickSight\_64.pdf} {{\textbackslash}AQuickSight} {QuickSight}

\gxs{\href{https://www.google.com/search?q=AWS+RDS}{RDS} \index{RDS}} {\ARDS{\iconsize}} {Arch\_Amazon-RDS\_64.pdf} {{\textbackslash}ARDS} {RDS}

\gxs{\href{https://www.google.com/search?q=AWS+RDS+for+VMware}{RDS for VMware} \index{RDS for VMware}} {\ARDSforVMware{\iconsize}} {Arch\_Amazon-RDS-for-VMware\_64.pdf} {{\textbackslash}ARDSforVMware} {RDS for VMware}

\gxs{\href{https://www.google.com/search?q=AWS+Redshift}{Redshift} \index{Redshift}} {\ARedshift{\iconsize}} {Arch\_Amazon-Redshift\_64.pdf} {{\textbackslash}ARedshift} {Redshift}

\gxs{\href{https://www.google.com/search?q=AWS+Redshiftct}{Redshiftct} \index{Redshiftct}} {\ARedshiftct{\iconsize}} {Arch\_Amazon-Redshiftct\_64.pdf} {{\textbackslash}ARedshiftct} {Redshiftct}

\gxs{\href{https://www.google.com/search?q=AWS+Rekognition}{Rekognition} \index{Rekognition}} {\ARekognition{\iconsize}} {Arch\_Amazon-Rekognition\_64.pdf} {{\textbackslash}ARekognition} {Rekognition}

\gxs{\href{https://www.google.com/search?q=AWS+Reserved+Instance+Reporting}{Reserved Instance Reporting} \index{Reserved Instance Reporting}} {\AReservedInstReporting{\iconsize}} {Arch\_Reserved-Instance-Reporting\_64.pdf} {{\textbackslash}AReservedInstReporting} {Reserved Instance Reporting}

\gxs{\href{https://www.google.com/search?q=AWS+Resources+Access+Manager}{Resources Access Manager} \index{Resources Access Manager}} {\ARessAccessMngr{\iconsize}} {Arch\_AWS-Resources-Access-Manager\_64.pdf} {{\textbackslash}ARessAccessMngr} {Resources Access Manager}

\gxs{\href{https://www.google.com/search?q=AWS+RoboMaker}{RoboMaker} \index{RoboMaker}} {\ARoboMaker{\iconsize}} {Arch\_AWS-RoboMaker\_64.pdf} {{\textbackslash}ARoboMaker} {RoboMaker}

\gxs{\href{https://www.google.com/search?q=AWS+Route+53}{Route 53} \index{Route 53}} {\ARouteFiveThree{\iconsize}} {Arch\_Amazon-Route-53\_64.pdf} {{\textbackslash}ARouteFiveThree} {Route 53}

\gxs{\href{https://www.google.com/search?q=AWS+S3+Standard}{S3 Standard} \index{S3 Standard}} {\ASThreeStandard{\iconsize}} {Arch\_Amazon-S3-Standard\_64.pdf} {{\textbackslash}ASThreeStandard} {S3 Standard}

\gxs{\href{https://www.google.com/search?q=AWS+SageMaker}{SageMaker} \index{SageMaker}} {\ASageMaker{\iconsize}} {Arch\_AWS-SageMaker\_64.pdf} {{\textbackslash}ASageMaker} {SageMaker}

\gxs{\href{https://www.google.com/search?q=AWS+SageMaker+Ground+Truth}{SageMaker Ground Truth} \index{SageMaker Ground Truth}} {\ASageMakerGroundTruth{\iconsize}} {Arch\_AWS-SageMaker-Ground-Truth\_64.pdf} {{\textbackslash}ASageMakerGroundTruth} {SageMaker Ground Truth}

\gxs{\href{https://www.google.com/search?q=AWS+Savings+Plans}{Savings Plans} \index{Savings Plans}} {\ASavingsPlans{\iconsize}} {Arch\_Savings-Plans\_64.pdf} {{\textbackslash}ASavingsPlans} {Savings Plans}

\gxs{\href{https://www.google.com/search?q=AWS+Secrets+Manager}{Secrets Manager} \index{Secrets Manager}} {\ASecretsMngr{\iconsize}} {Arch\_AWS-Secrets-Manager\_64.pdf} {{\textbackslash}ASecretsMngr} {Secrets Manager}

\gxs{\href{https://www.google.com/search?q=AWS+Security+Hub}{Security Hub} \index{Security Hub}} {\ASecHub{\iconsize}} {Arch\_AWS-Security-Hub\_64.pdf} {{\textbackslash}ASecHub} {Security Hub}

\gxs{\href{https://www.google.com/search?q=AWS+Server+Migration+Service}{Server Migration Service} \index{Server Migration Service}} {\AServerMigratSvc{\iconsize}} {Arch\_AWS-Server-Migration-Service\_64.pdf} {{\textbackslash}AServerMigratSvc} {Server Migration Service}

\gxs{\href{https://www.google.com/search?q=AWS+Serverless+Application+Repository}{Serverless Application Repository} \index{Serverless Application Repository}} {\ASvlsAppRepo{\iconsize}} {Arch\_AWS-Serverless-Application-Repository\_64.pdf} {{\textbackslash}ASvlsAppRepo} {Serverless Application Repository}

\gxs{\href{https://www.google.com/search?q=AWS+Service+Catalog}{Service Catalog} \index{Service Catalog}} {\ASvcCatalog{\iconsize}} {Arch\_AWS-Service-Catalog\_64.pdf} {{\textbackslash}ASvcCatalog} {Service Catalog}

\gxs{\href{https://www.google.com/search?q=AWS+Shield}{Shield} \index{Shield}} {\AShield{\iconsize}} {Arch\_AWS-Shield\_64.pdf} {{\textbackslash}AShield} {Shield}

\gxs{\href{https://www.google.com/search?q=AWS+Simple+Email+Service}{Simple Email Service} \index{Simple Email Service}} {\ASimpleEmailSvc{\iconsize}} {Arch\_AWS-Simple-Email-Service\_64.pdf} {{\textbackslash}ASimpleEmailSvc} {Simple Email Service}

\gxs{\href{https://www.google.com/search?q=AWS+Simple+Notification+Service}{Simple Notification Service} \index{Simple Notification Service}} {\ASimpleNotifSvc{\iconsize}} {Arch\_AWS-Simple-Notification-Service\_64.pdf} {{\textbackslash}ASimpleNotifSvc} {Simple Notification Service}

\gxs{\href{https://www.google.com/search?q=AWS+Simple+Queue+Service}{Simple Queue Service} \index{Simple Queue Service}} {\ASQS{\iconsize}} {Arch\_AWS-Simple-Queue-Service\_64.pdf} {{\textbackslash}ASQS} {Simple Queue Service}

\gxs{\href{https://www.google.com/search?q=AWS+Single+Sign+On}{Single Sign On} \index{Single Sign On}} {\ASingleSignOn{\iconsize}} {Arch\_AWS-Single-Sign-On\_64.pdf} {{\textbackslash}ASingleSignOn} {Single Sign On}

\gxs{\href{https://www.google.com/search?q=AWS+Site+to+Site+VPN}{Site to Site VPN} \index{Site to Site VPN}} {\ASitetoSiteVPN{\iconsize}} {Arch\_AWS-Site-to-Site-VPN\_64.pdf} {{\textbackslash}ASitetoSiteVPN} {Site to Site VPN}

\gxs{\href{https://www.google.com/search?q=AWS+Snowball}{Snowball} \index{Snowball}} {\ASnowball{\iconsize}} {Arch\_AWS-Snowball\_64.pdf} {{\textbackslash}ASnowball} {Snowball}

\gxs{\href{https://www.google.com/search?q=AWS+Snowball+Edge}{Snowball Edge} \index{Snowball Edge}} {\ASnowballEdge{\iconsize}} {Arch\_AWS-Snowball-Edge\_64.pdf} {{\textbackslash}ASnowballEdge} {Snowball Edge}

\gxs{\href{https://www.google.com/search?q=AWS+Snowcone}{Snowcone} \index{Snowcone}} {\ASnowcone{\iconsize}} {Arch\_AWS-Snowcone\_64.pdf} {{\textbackslash}ASnowcone} {Snowcone}

\gxs{\href{https://www.google.com/search?q=AWS+Snowmobile}{Snowmobile} \index{Snowmobile}} {\ASnowmobile{\iconsize}} {Arch\_AWS-Snowmobile\_64.pdf} {{\textbackslash}ASnowmobile} {Snowmobile}

\gxs{\href{https://www.google.com/search?q=AWS+Standard+Storage+Class}{Standard Storage Class} \index{Standard Storage Class}} {\AStandardStorageClass{\iconsize}} {Arch\_Standard-Storage-Class\_64.pdf} {{\textbackslash}AStandardStorageClass} {Standard Storage Class}

\gxs{\href{https://www.google.com/search?q=AWS+Step+Functions}{Step Functions} \index{Step Functions}} {\AStepFuncs{\iconsize}} {Arch\_AWS-Step-Functions\_64.pdf} {{\textbackslash}AStepFuncs} {Step Functions}

\gxs{\href{https://www.google.com/search?q=AWS+Storage+Gateway}{Storage Gateway} \index{Storage Gateway}} {\AStorGat{\iconsize}} {Arch\_AWS-Storage-Gateway\_64.pdf} {{\textbackslash}AStorGat} {Storage Gateway}

\gxs{\href{https://www.google.com/search?q=AWS+Sumerian}{Sumerian} \index{Sumerian}} {\ASumerian{\iconsize}} {Arch\_AWS-Sumerian\_64.pdf} {{\textbackslash}ASumerian} {Sumerian}

\gxs{\href{https://www.google.com/search?q=AWS+Support}{Support} \index{Support}} {\ASupport{\iconsize}} {Arch\_AWS-Support\_64.pdf} {{\textbackslash}ASupport} {Support}

\gxs{\href{https://www.google.com/search?q=AWS+Systems+Manager}{Systems Manager} \index{Systems Manager}} {\ASystemsMngr{\iconsize}} {Arch\_AWS-Systems-Manager\_64.pdf} {{\textbackslash}ASystemsMngr} {Systems Manager}

\gxs{\href{https://www.google.com/search?q=AWS+TensorFlow+on+AWS}{TensorFlow on AWS} \index{TensorFlow on AWS}} {\ATensorFlowon{\iconsize}} {Arch\_AWS-TensorFlow-on-AWS\_64.pdf} {{\textbackslash}ATensorFlowon} {TensorFlow on AWS}

\gxs{\href{https://www.google.com/search?q=AWS+Textract}{Textract} \index{Textract}} {\ATextract{\iconsize}} {Arch\_AWS-Textract\_64.pdf} {{\textbackslash}ATextract} {Textract}

\gxs{\href{https://www.google.com/search?q=AWS+ThinkBox+Deadline}{ThinkBox Deadline} \index{ThinkBox Deadline}} {\ATBDeadline{\iconsize}} {Arch\_AWS-ThinkBox-Deadline\_64.pdf} {{\textbackslash}ATBDeadline} {ThinkBox Deadline}

\gxs{\href{https://www.google.com/search?q=AWS+ThinkBox+Frost}{ThinkBox Frost} \index{ThinkBox Frost}} {\ATBFrost{\iconsize}} {Arch\_AWS-ThinkBox-Frost\_64.pdf} {{\textbackslash}ATBFrost} {ThinkBox Frost}

\gxs{\href{https://www.google.com/search?q=AWS+ThinkBox+Krakatoa}{ThinkBox Krakatoa} \index{ThinkBox Krakatoa}} {\ATBKrakatoa{\iconsize}} {Arch\_AWS-ThinkBox-Krakatoa\_64.pdf} {{\textbackslash}ATBKrakatoa} {ThinkBox Krakatoa}

\gxs{\href{https://www.google.com/search?q=AWS+ThinkBox+Sequoia}{ThinkBox Sequoia} \index{ThinkBox Sequoia}} {\ATBSequoia{\iconsize}} {Arch\_AWS-ThinkBox-Sequoia\_64.pdf} {{\textbackslash}ATBSequoia} {ThinkBox Sequoia}

\gxs{\href{https://www.google.com/search?q=AWS+ThinkBox+Stoke}{ThinkBox Stoke} \index{ThinkBox Stoke}} {\ATBStoke{\iconsize}} {Arch\_AWS-ThinkBox-Stoke\_64.pdf} {{\textbackslash}ATBStoke} {ThinkBox Stoke}

\gxs{\href{https://www.google.com/search?q=AWS+ThinkBox+XMesh}{ThinkBox XMesh} \index{ThinkBox XMesh}} {\ATBXMesh{\iconsize}} {Arch\_AWS-ThinkBox-XMesh\_64.pdf} {{\textbackslash}ATBXMesh} {ThinkBox XMesh}

\gxs{\href{https://www.google.com/search?q=AWS+Timestream}{Timestream} \index{Timestream}} {\ATimestream{\iconsize}} {Arch\_Amazon-Timestream\_64.pdf} {{\textbackslash}ATimestream} {Timestream}

\gxs{\href{https://www.google.com/search?q=AWS+Tools+and+SDK}{Tools and SDK} \index{Tools and SDK}} {\AToolsandSDK{\iconsize}} {Arch\_AWS-Tools-and-SDK\_64.pdf} {{\textbackslash}AToolsandSDK} {Tools and SDK}

\gxs{\href{https://www.google.com/search?q=AWS+TorchServe}{TorchServe} \index{TorchServe}} {\ATorchServe{\iconsize}} {Arch\_TorchServe\_64.pdf} {{\textbackslash}ATorchServe} {TorchServe}

\gxs{\href{https://www.google.com/search?q=AWS+Training+Certification}{Training Certification} \index{Training Certification}} {\ATrainCert{\iconsize}} {Arch\_AWS-Training-Certification\_64.pdf} {{\textbackslash}ATrainCert} {Training Certification}

\gxs{\href{https://www.google.com/search?q=AWS+Transcribe}{Transcribe} \index{Transcribe}} {\ATranscribe{\iconsize}} {Arch\_Amazon-Transcribe\_64.pdf} {{\textbackslash}ATranscribe} {Transcribe}

\gxs{\href{https://www.google.com/search?q=AWS+Transfer+Family}{Transfer Family} \index{Transfer Family}} {\ATranFam{\iconsize}} {Arch\_AWS-Transfer-Family\_64.pdf} {{\textbackslash}ATranFam} {Transfer Family}

\gxs{\href{https://www.google.com/search?q=AWS+Transit+Gateway}{Transit Gateway} \index{Transit Gateway}} {\ATransitGateway{\iconsize}} {Arch\_AWS-Transit-Gateway\_64.pdf} {{\textbackslash}ATransitGateway} {Transit Gateway}

\gxs{\href{https://www.google.com/search?q=AWS+Translate}{Translate} \index{Translate}} {\ATranslate{\iconsize}} {Arch\_Amazon-Translate\_64.pdf} {{\textbackslash}ATranslate} {Translate}

\gxs{\href{https://www.google.com/search?q=AWS+Trusted+Advisor}{Trusted Advisor} \index{Trusted Advisor}} {\ATrusTAdv{\iconsize}} {Arch\_AWS-Trusted-Advisor\_64.pdf} {{\textbackslash}ATrusTAdv} {Trusted Advisor}

\gxs{\href{https://www.google.com/search?q=AWS+VMware+Cloud+on+AWS}{VMware Cloud on AWS} \index{VMware Cloud on AWS}} {\AVMwareCloudon{\iconsize}} {Arch\_VMware-Cloud-on-AWS\_64.pdf} {{\textbackslash}AVMwareCloudon} {VMware Cloud on AWS}

\gxs{\href{https://www.google.com/search?q=AWS+VPN}{VPN} \index{VPN}} {\AVPN{\iconsize}} {Arch\_AWS-VPN\_64.pdf} {{\textbackslash}AVPN} {VPN}

\gxs{\href{https://www.google.com/search?q=AWS+WAF}{WAF} \index{WAF}} {\AWAF{\iconsize}} {Arch\_AWS-WAF\_64.pdf} {{\textbackslash}AWAF} {WAF}

\gxs{\href{https://www.google.com/search?q=AWS+Wavelength}{Wavelength} \index{Wavelength}} {\AWavelength{\iconsize}} {Arch\_AWS-Wavelength\_64.pdf} {{\textbackslash}AWavelength} {Wavelength}

\gxs{\href{https://www.google.com/search?q=AWS+Well+Architect+Tool}{Well Architect Tool} \index{Well Architect Tool}} {\AWellArchitectTool{\iconsize}} {Arch\_AWS-Well-Architect-Tool\_64.pdf} {{\textbackslash}AWellArchitectTool} {Well Architect Tool}

\gxs{\href{https://www.google.com/search?q=AWS+WorkDocs}{WorkDocs} \index{WorkDocs}} {\AWorkDocs{\iconsize}} {Arch\_Amazon-WorkDocs\_64.pdf} {{\textbackslash}AWorkDocs} {WorkDocs}

\gxs{\href{https://www.google.com/search?q=AWS+WorkLink}{WorkLink} \index{WorkLink}} {\AWorkLink{\iconsize}} {Arch\_AWS-WorkLink\_64.pdf} {{\textbackslash}AWorkLink} {WorkLink}

\gxs{\href{https://www.google.com/search?q=AWS+WorkMail}{WorkMail} \index{WorkMail}} {\AWorkMail{\iconsize}} {Arch\_Amazon-WorkMail\_64.pdf} {{\textbackslash}AWorkMail} {WorkMail}

\gxs{\href{https://www.google.com/search?q=AWS+WorkSpaces}{WorkSpaces} \index{WorkSpaces}} {\AWorkSpaces{\iconsize}} {Arch\_AWS-WorkSpaces\_64.pdf} {{\textbackslash}AWorkSpaces} {WorkSpaces}

\gxs{\href{https://www.google.com/search?q=AWS+X+Ray}{X Ray} \index{X Ray}} {\AXRay{\iconsize}} {Arch\_AWS-X-Ray\_64.pdf} {{\textbackslash}AXRay} {X Ray}

\archEnd



%%%%%%%%%%%%% INDEXES
\normalsize

\printindex
\printindex[macros]

\end{document}
